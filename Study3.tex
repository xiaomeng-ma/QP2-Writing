\section{Can Parents' Input Explain Pronoun Case Errors?}

\subsection{Ambiguity in parents' pronoun input}
Parents' input plays an important role in various aspects in children's language acquisition, such as verb agreement, \textit{be} and \textit{have} auxiliaries and auxiliary-verb inversion \citep[e.g.][]{rowland2000subject,theakston2001role,theakston2002going,theakston2005acquisition}. Many researchers also propose that children's pronoun errors stem from the ambiguous input in parents' utterances instead of innate syntactic knowledge of case assignment. \cite{pelham2011input} proposes that English-speaking children make more pronoun case errors than German-speaking children because there are more case-ambiguous pronouns, such as \textit{you} and \textit{it} in English parents' input. In addition, the ambiguous context for pronouns could also confuse the children. \cite{tomasello2000,tomasello2003} proposed that children make errors like `\textit{Her} open it' because they hear parents say `Let \textit{her} open it', and `they may just imitatively learn the end part of the sentence.' He further explains that children almost never make errors like `Mary likes \textit{I}.' because such sequence never occurs in parents' input. \cite{wexler2003lenneberg} argues two points against Tomasello's proposal. First, children do hear sequences where a nominative pronoun follows a verb, such as `Mary knows I like candy'; however, they rarely make `NOM-for-ACC' errors. Second, in German and Dutch, the parents' input are similar to English that accusative cases can appear before a verb such as `Mary saw him go', yet German and Dutch children do not use accusative case as the subject. Both \cite{tomasello2000,tomasello2003}'s proposal and \cite{wexler2003lenneberg}'s rebuttal sound reasonable; however, neither of them referenced any data to support their arguments. 

\cite{kirjavainen2009can} tested \cite{tomasello2000,tomasello2003}'s hypothesis on first person pronouns. They conducted a corpus analysis to investigate if utterances like `Let \textit{me} do it' in parents' input could explain `\textit{me} do it' errors in children's utterances. 17 children from 3 corpora (12 children from the Manchester corpus \citep{theakston2001}, 2 children from the dense Manchester corpus \citep{lieven2009two} and Abe from \cite{kuczaj1977acquisition}, Peter from \cite{bloom1974imitation} and Nina from \cite{suppes1974semantics}) were included in the study. To ensure that each child in the study were at similar developmental stage, they included 10 files for each child with MLU $\geq$ 2. They searched for all `\textit{me} + V' errors as well as `\textit{I} + V' and `\textit{my} + V' sequences in children's utterances. `\textit{I} + V' sequences include contracted forms such as \textit{I'm}, \textit{I've} or \textit{I'll}. The proportion of children's `\textit{me} + V' errors is calculated as $\displaystyle\frac{\text{`\textit{me} + V' error count}}{\text{`\textit{I} + V' count + `\textit{my} + V' count}}$. For parents' input, they searched 5 files preceding the children's files for `\textit{me} + V' sequence as well as `\textit{I} + V' and `\textit{my} + v' sequences. The `\textit{me} + V' sequences included non-finite complement clauses such as `\textit{help \underline{me pick} it up}' as well as all other instances where \textit{`me'} precedes a verb, including finite complement \textit{Show \underline{me is} there any dark left}, tag questions \textit{you gave it to \underline{me, didn't} you?} and two independent sentences \textit{you gave it to \underline{me. Have} you seen it?}. Parents' proportion of `\textit{me} + V' sequence is also calculated in the same as the children. The total numbers  of `\textit{me + V}' and `\textit{my + V}' errors and `\textit{I + V}' in 17 children's files are reproduced in Table \ref{tab:kir}. 4 children (Brian, Dominic, Fraser and John) didn't make any `\textit{me + V}' errors.

\FloatBarrier
\begin{table}[]
\centering
\caption{Summary of 1sg pronoun + V for 17 children in \cite{kirjavainen2009can}}
\label{tab:kir}
\begin{tabular}{llll}
\toprule
 & \multicolumn{1}{l}{me + V} & \multicolumn{1}{l}{I + V} & \multicolumn{1}{l}{my + V} \\
 \hline
Abe & 3 & 164 & 0 \\
Anne & 18 & 275 & 2 \\
Aran & 1 & 653 & 1 \\
Becky & 3 & 691 & 2 \\
Carl & 1 & 139 & 5 \\
Dominic & 0 & 523 & 3 \\
Fraser & 0 & 68 & 1 \\
Gail & 3 & 229 & 13 \\
Joel & 4 & 345 & 2 \\
John & 0 & 229 & 0 \\
Liz & 4 & 601 & 5 \\
Nicole & 2 & 132 & 3 \\
Nina & 4 & 458 & 45 \\
Peter & 30 & 1332 & 22 \\
Ruth & 211 & 124 & 9 \\
Warren & 5 & 219 & 13 \\
Brian & 0 & 5 & 0 \\
\bottomrule
\end{tabular}
\end{table}
\FloatBarrier


A Spearman's rho test showed that there is significant correlation between children's proportion of `\textit{me} + V' errors and parents' proportion of non-finite `\textit{me} + V' sequences (r_s = 0.55, p = 0.02), as well as all `\textit{me} + V' sequnces (r_s = 0.52, p = 0.03) . In addition, they also found that the proportion of `me' errors in children's utterances is not correlated with the proportion of `me' in parents' input, suggesting that a simply input-based theory couldn't explain this error. Children's pronoun case error rate is not correlated with parents' MLU, indicating the complexity of parents' input doesn't affect children's pronoun case errors. 

\subsection{Can `Let \textit{me/him/her/them} do it' explain `\textit{me/him/her/them}' do it error?}

\cite{kirjavainen2009can} only tested \cite{tomasello2000,tomasello2003}'s hypothesis on first person pronoun. It is still unclear whether this hypothesis can be applied to explain errors in third person pronoun. In addition, children also produce many correct `ACC + V' sequences such as `see \textit{him} cry'\footnote{\href{https://sla.talkbank.org/TBB/childes/Eng-NA/Brown/Adam/030707.cha}{Brown/Adam/030707.cha}} along with `ACC + V' errors. If children simply imitate the end part of parents' utterances, why would they be able to produce the correct `ACC + V' and incorrect `ACC + V' at the same time? In order to further test \cite{tomasello2000,tomasello2003}'s hypothesis, data from 46 longitudinal children in CHILDES were used to replicate \cite{kirjavainen2009can} on first person and third person pronouns.

Instead of using 10 files for children's data and 5 files for parents' data, all the files were used to search for children's and parents' proportional use of `ACC + V'. For children's data, both correct use of `\textit{me/her/him/them} + V' sequence (e.g. `see \underline{\textit{him} cry}') and error `\textit{me/her/him/them} + V' (e.g. `\textit{them} do it') were counted, as well as `\textit{I/she/he/they} + V' sequence. For parents' data, `\textit{me/her/him/them} + V' sequence and `\textit{I/she/he/they} + V' sequence were counted. The proportional use of `ACC + V' is calculated the same way in \cite{kirjavainen2009can}:
\begin{align*}
    \text{Proportion of correct `ACC + V'} = \displaystyle\frac{\text{correct `ACC + V' count}}{\text{correct and error `ACC + V' counts +  `NOM + V' count}}
\end{align*}
\begin{align*}
    \text{Proportion of error `ACC + V'} = \displaystyle\frac{\text{error `ACC + V' count}}{\text{correct and error `ACC + V' counts +  `NOM + V' count}}
\end{align*}
\begin{align*}
    \text{Proportion of input `ACC + V'} = \displaystyle\frac{\text{parents' `ACC + V' count}}{\text{parents' `ACC + V' counts + parents' `NOM + V' count}}
\end{align*}
Not all children made `ACC + V' errors: 41 children made `\textit{me} + V' errors; 27 children made `\textit{her} + V' errors; 20 children made `\textit{him} + V' errors and 26 children made `\textit{them} + V' errors. Only 8 children made errors on all four accusative pronouns. In addition, most of the children made `ACC + V' errors as well as produced correct `ACC + V' sequences. All 41 children who made `\textit{me} + V' errors also produced correct `\textit{me} + V' sequences; 18 out of the 27 children who made `\textit{her} + V' errors also produced correct `\textit{her} + V' sequences; 14 out of the 20 children who made made `\textit{him} + V' errors also produced correct `\textit{him} + V' sequences and 20 out of the 26 children made \textit{`them} + V' errors also produced correct `\textit{them} + V' sequences. In addition, error `\textit{her} + V' has the highest average proportion, 7.1\%; whereas the average proportion of error `\textit{me} + V', `\textit{him} + V' and `\textit{them} + V' is around 1\%. The correct `\textit{me} + V' sequence has the highest average proportion which is 3.5\% and the average proportions of the correct `\textit{her} + V', `\textit{him} + V' and  `\textit{them} + V' are around 1\%. The average proportions of parents' `ACC + V' sequences are generally higher than the average proportions of children's correct `ACC + V' sequences. The average proportions for each pronoun and the total tokens of children's and parents' `NOM + V' are shown in Table \ref{table:longtable}. The number of children's error and correct `ACC + V' sequences and parents' `ACC + V' sequences are shown in Table \ref{tab:kkkiiirrr}. 

\FloatBarrier
\begin{table}[!h]
\centering
\caption{Summary of `ACC + V' data for four accusative pronouns}
\label{table:longtable}
\begin{tabular}{l|llllll}
\hline
 & \begin{tabular}[c]{@{}l@{}}No. of children\\ making error\end{tabular} & \begin{tabular}[c]{@{}l@{}}Error \\ `ACC + V' \\ proportion\end{tabular} &
 \begin{tabular}[c]{@{}l@{}}Correct \\ `ACC + V' \\ proportion\end{tabular} &
 \begin{tabular}[c]{@{}l@{}}Parents'\\ `ACC + V' \\ proportion\end{tabular} & \begin{tabular}[c]{@{}l@{}}Child\\ `NOM + V' \\ tokens\end{tabular} & \begin{tabular}[c]{@{}l@{}}Parents' \\ `NOM + V'\\ tokens\end{tabular} \\ \hline
‘me + V’ & 41 & 0.009 & 0.035 & 0.073 & 94866 & 71824 \\
‘him + V’ & 20 & 0.007 & 0.006 & 0.027 & 12122 & 18497 \\
‘her + V’ & 27 & 0.071 & 0.07 & 0.046 & 3698 & 14738 \\
‘them + V’ & 26 & 0.013 & 0.010 & 0.083 & 6687 & 13889 \\ 
all `ACC + V' & 9 & 0.011 & 0.024 & 0.070 & 117373 & 118948\\
\hline

\end{tabular}
\end{table}
\FloatBarrier

The correlation tests were conducted on each accusative pronouns between parents' `ACC + V' proportions and children's error `ACC + V' proportions, parents' `ACC + V' proportions and children's correct `ACC + V' proportions, and children's correct and error  `ACC + V' proportions. As shown in Table \ref{tab:correACC}, only a weak positive correlation was found between parents' `\textit{her}+ V' proportion and children's error `\textit{her} + V'. For \textit{me}, \textit{him}, \textit{them} and the sum of accusative pronouns, parents' `ACC + V' proportions are not significantly correlated with children's error `ACC + V' proportions. For children's correct and error `ACC + V' proportions, there was a weak significant positive correlation on `\textit{her} + V', but on other accusative pronouns or the sum of accusative pronouns. In addition, no significant correlation were found between parents' `ACC + V' proportions and children's correct `ACC + V' proportions. The correlation results suggested that children's `ACC + V' errors are not correlated with parents' `ACC + V' input or children's own correct `ACC + V' uses. In addition, children's correct `ACC + V' uses are not correlated with parents' input either. 

The significant correlation between parents' `\textit{me} + V' proportions and children error `\textit{me} + V' proportions in \cite{kirjavainen2009can} was not replicated with 46 children's data. In \cite{kirjavainen2009can}, they used 10 files for each child; while in this paper, all the files were used to search for `\textit{me} + V' errors. In order to examine if the two datasets were comparable, correlations between the number of `\textit{me} + V' errors and proportions in two datasets for 16 common children were calculated. As shown in Table \ref{kvsthisp}, the number of `\textit{me} + V' errors in both datasets were highly correlated (r = 0.995), and so were the proprotions of `\textit{me} + V' errors (r = 0.996). This result suggests that the data from all the files in a child is comparable to 10 files for a child, which implies that the different data selection criterion should not be the reason why the positive correlation in \cite{kirjavainen2009can} was not replicated in this study. 
\FloatBarrier
\begin{table}[!h]
\centering
\caption{Numbers and proprotions of `\textit{me} + V' errors in \cite{kirjavainen2009can} and this paper }
\label{kvsthisp}
 \begin{tabular}{l|l|l|l|l}
 \toprule
 & \multicolumn{2}{l|}{Number of `\textit{me} + V errors'} & \multicolumn{2}{l}{Proportions of `\textit{me} + V' errors} \\ \hline
 & \begin{tabular}[c]{@{}l@{}}Data in Kirjavainen\\  et al. (2009)\end{tabular} & \begin{tabular}[c]{@{}l@{}}Data in\\ this paper\end{tabular} & \begin{tabular}[c]{@{}l@{}}Data in Kirjavainen\\  et al. (2009)\end{tabular} & \begin{tabular}[c]{@{}l@{}}Data in \\ this paper\end{tabular} \\ \hline
Abe & 3 & 20 & 0.018 & 0.003 \\
Anne & 18 & 25 & 0.061 & 0.029 \\
Aran & 1 & 6 & 0.002 & 0.003 \\
Becky & 3 & 5 & 0.004 & 0.003 \\
Carl & 1 & 6 & 0.007 & 0.003 \\
Dominic & 0 & 8 & 0.000 & 0.005 \\
Fraser & 0 & 22 & 0.000 & 0.002 \\
Gail & 3 & 1 & 0.012 & 0.001 \\
Joel & 4 & 8 & 0.011 & 0.006 \\
John & 0 & 4 & 0.000 & 0.006 \\
Liz & 4 & 15 & 0.007 & 0.010 \\
Nicole & 2 & 18 & 0.015 & 0.043 \\
Nina & 4 & 12 & 0.008 & 0.005 \\
Peter & 30 & 38 & 0.022 & 0.021 \\
Ruth & 211 & 425 & 0.613 & 0.489 \\
Warren & 5 & 8 & 0.021 & 0.007 \\
\hline
Correlation & \multicolumn{2}{l|}{r = 0.995} & \multicolumn{2}{l}{r = 0.996}\\
\bottomrule
\end{tabular}
\end{table}
\FloatBarrier

\FloatBarrier
\begin{table}[!h]
\centering
\caption{Correlations between the parents' `ACC + V' proportion, children's correct and error `ACC + V' proportions }
\label{tab:correACC}
  \begin{tabular}{cccc}
 \toprule
\multicolumn{1}{l|}{\begin{tabular}[c]{@{}l@{}}Correlation\\ (n = 46)\end{tabular}} & \begin{tabular}[c]{@{}l@{}}Parents' `ACC + V' VS & Error `ACC + V'\end{tabular} & \begin{tabular}[c]{@{}l@{}}Parents' `ACC + V' VS & Correct `ACC + V'\end{tabular} & \begin{tabular}[c]{@{}l@{}} Error `ACC + V' VS & Correct `ACC + V' \end{tabular} \\ \hline
\multicolumn{1}{l|}{‘me + V’} & 0.10 (p = 0.43) & 0.07 (p = 0.62) & 0.11 (p = 0.45) \\
\multicolumn{1}{l|}{‘him + V’} & 0.07 (p = 0.75) & -0.25 (p = 0.10) & -0.12 (p = 0.51)\\
\multicolumn{1}{l|}{‘her + V’} & \textbf{0.38*} (p = 0.01) & -0.06 (p = 0.74) & \textbf{0.31*} (p = 0.03)\\
\multicolumn{1}{l|}{‘them + V’} & 0.04 (p = 0.78) & 0.05 (p = 0.74) & 0.23 (p = 0.13)\\ 
\multicolumn{1}{l|}{all ‘ACC + V’} & 0.09 (p = 0.54) & -0.00 (p = 0.98) & 0.13 (p = 0.39)\\
\bottomrule
\end{tabular}
\end{table}
\FloatBarrier

Since not all the children made ‘ACC + V’ errors, Spearman's test was conducted to calculated the correlation limited to the children who made more than 1 `ACC + V' errors. As shown in Table \ref{tab:correACC22}, still no significant correlation was found. These results further demonstrated that children's `ACC + V' errors are not correlated with parents' `ACC + V' sequences. 
\FloatBarrier
\begin{table}[!h]
\centering
\caption{Correlations between the parents' `ACC + V' proportion, children's correct and error `ACC + V' proportions of the children who made more than 1 errors}
\label{tab:correACC22}
  \begin{tabular}{l|cccc}
 \toprule
\multicolumn{1}{l|}{\begin{tabular}[c]{@{}l@{}}Correlation\\ (Spearman's \\rho)\end{tabular}} & {\begin{tabular}[c]{@{}c@{}}N\\ (children \\who made \\$\geq$ 1 errors)\end{tabular}}   & \begin{tabular}[c]{@{}l@{}}Parents' `ACC + V' VS & Error `ACC + V'\end{tabular} & \begin{tabular}[c]{@{}l@{}}Parents' `ACC + V' VS & Correct `ACC + V'\end{tabular} & \begin{tabular}[c]{@{}l@{}} Error `ACC + V' VS & Correct `ACC + V' \end{tabular} \\ \hline
\multicolumn{1}{l|}{‘me + V’} & 34 &  0.26 (p = 0.14) & 0.06 (p = 0.76) & 0.06 (p = 0.74) \\
\multicolumn{1}{l|}{‘him + V’} & 12 & 0.38 (p = 0.23) & 0.04 (p = 0.90) & 0.11 (p = 0.74)\\
\multicolumn{1}{l|}{‘her + V’} & 21 & 0.25 (p = 0.28) & 0.22 (p = 0.34) & 0.13 (p = 0.57)\\
\multicolumn{1}{l|}{‘them + V’} & 17 & 0.30 (p = 0.24) & 0.05 (p = 0.86) & 0.16 (p = 0.55)\\ 
\bottomrule
\end{tabular}
\end{table}
\FloatBarrier

In addition, parents' `ACC + V' proportions were compared between the children who at least 2 errors and children who made 1 or 0 error. If children make `ACC + V' errors because of parents' input, parents' `ACC + V' proportions should be different for the children who make errors and who don't make errors. However, Table \ref{tab:ttestinput} shows that the means of the parents' `ACC + V' proportions are not different for children who made more than 1 errors and children who made 1 and 0 error.

Furthermore, more correlation tests were conducted to investigate if the `ACC + V' sequences on different pronouns are correlated. It is unclear if children perceive \textit{me}, \textit{her}, \textit{him}, and \textit{them} as individual pronouns, or as a whole category such as accusative pronoun. If they detect the commonalities in those accusative pronouns, the proportions of `ACC + V' for different pronouns are more likely to be correlated. The correlation matrices for four pronouns of children's error and correct `ACC + V' and parents' `ACC + V' proportions are shown in Table \ref{tab: matrices}. For children's error and correct `ACC + V' proportions, only error `\textit{him} + V' and error `\textit{her} + V' proportions are positively correlated and no significant correlations were found for other pairs. For parents' `ACC + V' proportions, only two pairs of pronouns, `me + V' and `them + V', `him + V' and `them + V' are not correlated and all other pairs of pronouns are significantly positively correlated. These correlation results suggest that children are more likely to treat different accusative pronouns as distinctive individual pronouns. 

\FloatBarrier
\begin{table}[!h]
\centering
\caption{Comparison of parents' `ACC + V' proportions between children who made more than 1 errors and who made 0 and 1 error}
\label{tab:ttestinput}
\begin{tabular}{l|ccc|ccc|cc}
\toprule
 & \multicolumn{3}{l|}{Children with $\geq$ 2 error} & \multicolumn{3}{l|}{Children with 0 and 1 error} &  &  \\ \hline
\begin{tabular}[c]{@{}l@{}}Input \\ Proportion\end{tabular} & N & \multicolumn{1}{l}{Mean} & SD & N & Mean & SD & t & p \\
\hline
`me + V' & 34 & 0.09 & 0.09 & 11 & 0.08 & 0.06 & 0.51 & 0.61 \\
`him + V' & 12 & 0.03 & 0.02 & 33 & 0.03 & 0.02 & 0.67 & 0.51 \\
`them + V' & 17 & 0.08 & 0.02 & 28 & 0.09 & 0.14 & -0.57 & 0.57 \\
`her + V' & 21 & 0.06 & 0.05 & 24 & 0.07 & 0.14 & -0.56 & 0.58\\
\bottomrule
\end{tabular}
\end{table}
\FloatBarrier

\FloatBarrier
\begin{table}[!h]
\centering
\caption{Correlation matrix for children's correct and error `ACC + V' and parents' `ACC + V' proportions of different pronouns }
\label{tab: matrices}
\begin{minipage}{0.8\textwidth}
\centering
\subcaption{Correlation of children's error `ACC + V' proportions among different pronouns}
\begin{tabular}{l|llll}
\toprule
\textbf{Children's Error} & `me + V' & ‘her + V' & ‘him + V’ & ‘them + V’ \\ \hline
`me + V'  & 1 & -0.08 & -0.11 & 0.16 \\
‘her + V’ &  & 1 & \textbf{0.43**} & 0.07 \\
‘him + V’ &  &  & 1 & -0.17 \\
‘them + V’ &  &  &  & 1\\
\bottomrule
\end{tabular}
\end{minipage}
\begin{minipage}{0.95\textwidth}
\centering
\subcaption{Correlation of children's correct `ACC + V' proportions among different pronouns}
\begin{tabular}{l|llll}
\toprule
\textbf{Children's Correct} & `me + V' & ‘her + V’ & ‘him + V’ & ‘them + V’ \\
\hline
`me + V' & 1.00 & -0.15 & 0.01 & 0.22 \\
‘her + V’&  & 1.00 & -0.17 & -0.11 \\
‘him + V’&  &  & 1.00 & -0.03 \\
‘them + V’ &  &  &  & 1.00\\
\bottomrule
\end{tabular}
\end{minipage}
\begin{minipage}{0.8\textwidth}
\centering
\subcaption{Correlation of parents' `ACC + V' proportions among different pronouns}
\begin{tabular}{l|llll}
\toprule
\textbf{Parents' Input} & `me + V' & ‘her + V’ & ‘him + V’ & ‘them + V’\\
\hline
`me + V'  & 1.00 & \textbf{0.58**} & \textbf{0.39**} & 0.07 \\
 ‘her + V’ &  & 1.00 & \textbf{0.34*} & \textbf{0.49***} \\
 ‘him + V’ &  &  & 1.00 & -0.25 \\
‘them + V’ &  &  &  & 1.00\\
\bottomrule
\end{tabular}

\end{minipage}


\end{table}
\FloatBarrier




Moreover, the verbs that were used in children's `ACC + V' errors were compared to the verbs used in parents' `ACC + V' sequences. If children made `ACC + V' errors by mistakenly imitating parents' input, there should be a high rate of overlap among these verbs. Different forms of the verb were counted as different types, e.g. \textit{do} and \textit{did} were counted as two verbs. The number of verbs in children's `ACC + V' errors and parents' `ACC + V' sequences, the number of over lap verbs and the overlap rate for each pronoun are shown in Tables \ref{tab: overlapme}, \ref{tab: overlapher}, \ref{tab: overlaphim} and \ref{tab: overlapthem}. 
For 34 children who made more than 1 `\textit{me} + V' errors, 11 children had no overlap verbs with parents' `\textit{me} + V' input. The overlap rate is calculated as the number of overlap verb types over the number of children's verb types. The average number of verb types for children's \textit{`me} + V' errors is 9.23, and for parents' `\textit{me} + V' input is 27.2. The average number of overlap verb types is 2.6, with an overlap rate of 27.1\%. Only one child, Matt, had 100\% overlap rate. He made `\textit{me} + V' errors on three different verbs (\textit{do, have, take}) and those verbs were found in his parents' `\textit{me} + V' sequences. The most common overlap verbs in children's' `\textit{me} + V' errors and parents' `\textit{me} + V' sequences are \textit{do}, \textit{go} and \textit{have}. The most frequent verb phrases containing the overlap verbs in parents' utterances are `let me V' and `help me V'. In addition, children's `\textit{me} + V' error not only appeared in declarative sentences such as `\textit{Me do‘} or `\textit{Now me go}’, `\textit{me} + V' has also been found in questions such as `\textit{Can me have it?}'\footnote{Liz at age 2;6, \href{https://childes.talkbank.org/browser/index.php?url=Eng-UK/Manchester/Liz/020605.cha}{Manchester/Liz/020605.cha}} and clauses such as `\textit{when me fall in the swimming pool}'\footnote{\textsuperscript{,12}Nina at age 2;5, \href{https://childes.talkbank.org/browser/index.php?url=Eng-NA/Suppes/020527.cha}{Suppes/020527.cha}} and  `\textit{a train me got}'\textsuperscript{12}. For 21 children who made more than 1 \textit{`her} + V' errors, 13 children had no overlap verbs with their parents' `\textit{her} + V' verbs. The average number of overlap verb is 0.52 and the overlap rate is 8.7\%. The most common overlap verb is \textit{go}. For 12 children who made more than 1 `\textit{him} + V' errors, only 2 children have overlap verbs with their parents' \textit{`him + V}' sequences. The average number of overlap verbs is 0.25 and the overlap rate is 6.5\%. For 17 children who made more than 1 errors with `\textit{them} + V', 9 children's verbs in `\textit{them} + V' errors overlapped with the verbs in their parents' `\textit{them} + V' sequences. The average number of overlap verb is 3 and the overlap rate is 30.9\%. These results show that most of the verbs appear in children's `ACC + V' errors do not overlap with the verbs in their parents' `ACC + V'. 

\FloatBarrier
\begin{table}[!h]
\centering
\caption{Overlap verbs in parents' `\textit{her} + V' sequences and children's `\textit{her} + V' errors }
\label{tab: overlapher}
\begin{tabular}{l|lllll}
\toprule
Child & \multicolumn{1}{l}{Verb types} & \multicolumn{1}{l}{\begin{tabular}[c]{@{}l@{}}Overlap \\ Verb types\end{tabular}} & \multicolumn{1}{l}{\begin{tabular}[c]{@{}l@{}}Overlap\\ Rate\end{tabular}} & \multicolumn{1}{l}{\begin{tabular}[c]{@{}l@{}}Parents'\\ Verb types\end{tabular}} & \begin{tabular}[c]{@{}l@{}}overlap\\ verbs\end{tabular} \\
\hline
Abe & 14 & 0 & 0.0\% & 5 &  \\
Ben & 12 & 0 & 0.0\% & 1 &  \\
Lara & 6 & 1 & 16.7\% & 37 & go \\
Ross & 7 & 0 & 0.0\% & 0 &  \\
Thomas & 3 & 0 & 0.0\% & 91 &  \\
Adam & 4 & 1 & 25.0\% & 8 & see \\
Nicole & 2 & 0 & 0.0\% & 9 &  \\
Anne & 5 & 2 & 40.0\% & 21 & cry, got \\
Nina & 15 & 1 & 6.7\% & 23 & sleeping, \\
Sarah & 14 & 3 & 21.4\% & 17 & do, go, make \\
Tow & 7 & 0 & 0.0\% & 15 &  \\
Aran & 2 & 1 & 50.0\% & 19 & go \\
Courtney & 2 & 0 & 0.0\% & 12 &  \\
Eve & 3 & 0 & 0.0\% & 1 &  \\
Matt & 3 & 0 & 0.0\% & 4 &  \\
She & 3 & 0 & 0.0\% & 13 &  \\
Fraser & 3 & 0 & 0.0\% & 20 &  \\
Rachel & 6 & 1 & 16.7\% & 6 & to \\
Becky & 3 & 1 & 33.3\% & 9 & is \\
Gail & 7 & 0 & 0.0\% & 14 &  \\
Naomi & 6 & 0 & 0.0\% & 1 &  \\
\hline
Average & 6.05 & 0.52 & 8.7\% & 15.52 & \\
\bottomrule
\end{tabular}
\end{table}
\FloatBarrier



\FloatBarrier
\begin{table}[!h]
\centering
\caption{Overlap verbs in parents' `\textit{me} + V' sequences and children's `\textit{me} + V' errors }
\label{tab: overlapme}
\begin{tabular}{l|lllll}
\toprule
Child & \multicolumn{1}{l}{Verb types} & \multicolumn{1}{l}{\begin{tabular}[c]{@{}l@{}}Overlap \\ Verb types\end{tabular}} & \multicolumn{1}{l}{\begin{tabular}[c]{@{}l@{}}Overlap\\ Rate\end{tabular}} & \multicolumn{1}{l}{\begin{tabular}[c]{@{}l@{}}Parents'\\ Verb types\end{tabular}} & \begin{tabular}[c]{@{}l@{}}Overlap\\ verbs\end{tabular} \\
\hline
Abe & 16 & 2 & 12.5\% & 23 & do,make, \\
Adam & 25 & 4 & 16.0\% & 20 & get, go, put, want \\
Anne & 14 & 6 & 42.9\% & 36 & \begin{tabular}[c]{@{}l@{}}are, do, get, have, \\ make, put\end{tabular} \\
Aran & 4 & 1 & 25.0\% & 34 & play \\
Barbara & 2 & 0 & 0.0\% & 25 &  \\
Becky & 4 & 1 & 25.0\% & 32 & do \\
Carl & 4 & 0 & 0.0\% & 18 &  \\
Dominic & 7 & 4 & 57.1\% & 34 & do,fix, have, see \\
Eleanor & 18 & 7 & 38.9\% & 39 & \begin{tabular}[c]{@{}l@{}}come, do, get, go, look, \\ open, to\end{tabular} \\
Emily & 3 & 0 & 0.0\% & 1 &  \\
Emma & 2 & 0 & 0.0\% & 3 &  \\
Eve & 3 & 0 & 0.0\% & 17 &  \\
Fraser & 20 & 6 & 30.0\% & 55 & can't, do, get, go, see, to \\
Jimmy & 3 & 0 & 0.0\% & 5 &  \\
Joel & 7 & 3 & 42.9\% & 27 & do, go, have \\
Lara & 12 & 7 & 58.3\% & 62 & \begin{tabular}[c]{@{}l@{}}do, eat, give, go, have, \\ turn, write,\end{tabular} \\
Laura & 13 & 4 & 30.8\% & 21 & do, eat, get, keep \\
Liz & 8 & 4 & 50.0\% & 19 & get, go, have, put \\
Matt & 3 & \textbf{3} & 100.0\% & 39 & do, have, take \\
Michelle & 2 & 0 & 0.0\% & 11 &  \\
Naomi & 13 & 2 & 15.4\% & 13 & go, put \\
Nathaniel & 2 & 0 & 0.0\% & 11 &  \\
Nicole & 14 & 4 & 28.6\% & 35 & get, go, have, can \\
Nina & 9 & 4 & 44.4\% & 35 & eat, give, go, have \\
Peter & 17 & 1 & 5.9\% & 6 & put \\
Ross & 14 & 1 & 7.1\% & 9 & do \\
Ruth & 27 & 0 & 0.0\% & 30 &  \\
Sarah & 5 & 2 & 40.0\% & 33 & see, show \\
She & 6 & 1 & 16.7\% & 17 & read \\
Shem & 2 & 0 & 0.0\% & 5 &  \\
Thomas & 21 & 14 & 66.7\% & 179 & \begin{tabular}[c]{@{}l@{}}do, don't, finish, get, go, got, \\ have, hold, open, read, shut,  \\ think, to, waiting,\end{tabular} \\
David & 2 & 0 & 0.0\% & 4 &  \\
Warren & 8 & 2 & 25.0\% & 17 & do, have \\
John & 4 & 2 & 50.0\% & 16 & do, have\\
\hline
Average & 9.23 & 2.6 & 27.1\%  & 27.2 & \\
\bottomrule
\end{tabular}
\end{table}
\FloatBarrier

\FloatBarrier
\begin{table}[!t]
\centering
\caption{Overlap verbs in parents' `\textit{him} + V' sequences and children's `\textit{him} + V' errors }
\label{tab: overlaphim}
\begin{tabular}{l|lllll}
\toprule
Child & \multicolumn{1}{l}{Verb types} & \multicolumn{1}{l}{\begin{tabular}[c]{@{}l@{}}Overlap \\ Verb types\end{tabular}} & \multicolumn{1}{l}{\begin{tabular}[c]{@{}l@{}}Overlap\\ Rate\end{tabular}} & \multicolumn{1}{l}{\begin{tabular}[c]{@{}l@{}}Parents'\\ Verb types\end{tabular}} & \begin{tabular}[c]{@{}l@{}}Overlap\\ verbs\end{tabular} \\
\hline
Ben & 2 & 0 & 0.0\% & 2& \\
Ross & 6 & 1 & 16.7\% & 10& do \\
Thomas & 2 & 0 & 0.0\% & 49&  \\
Nina & 13 & 2 & 15.4\% & 24&  go, eat \\
Sarah & 2 & 0 & 0.0\% & 8& \\
Matt & 5 & 0 & 0.0\% & 14& \\
Fraser & 2 & 0 & 0.0\% & 19& \\
Shem & 2 & 0 & 0.0\% & 2& \\
Becky & 4 & 0 & 0.0\% & 14&  \\
Gail & 3 & 0 & 0.0\% & 16 & \\
Naomi & 2 & 0 & 0.0\% & 1& \\
Roman & 3 & 0 & 0.0\% & 0& \\
\hline
Average & 3.83 & 0.25 & 6.5\% & 13.25 & \\
\bottomrule
\end{tabular}
\end{table}
\begin{table}[!b]
\centering
\caption{Overlap verbs in parents' `\textit{them} + V' sequences and children's `\textit{them} + V' errors }
\label{tab: overlapthem}
\begin{tabular}{l|lllll}
\toprule
Child & \multicolumn{1}{l}{Verb types} & \multicolumn{1}{l}{\begin{tabular}[c]{@{}l@{}}Overlap \\ Verb types\end{tabular}} & \multicolumn{1}{l}{\begin{tabular}[c]{@{}l@{}}Overlap\\ Rate\end{tabular}} & \multicolumn{1}{l}{\begin{tabular}[c]{@{}l@{}}Parents'\\ Verb types\end{tabular}} & \begin{tabular}[c]{@{}l@{}}Overlap\\ verbs\end{tabular} \\
\hline
Abe & 3 & 0 & 0.0\% & 6 &  \\
Eleanor & 4 & 0 & 0.0\% & 19 &  \\
Lara & 2 & 0 & 0.0\% & 34 &  \\
Thomas & 11 & 8 & 72.7\% & 141 & \begin{tabular}[c]{@{}l@{}}do, doing, go, has, \\ have, is, open, put,\end{tabular} \\
Laura & 3 & 0 & 0.0\% & 4 &  \\
Nicole & 3 & 1 & 33.3\% & 17 & can't \\
Jimmy & 2 & 0 & 0.0\% & 2 &  \\
Sarah & 5 & 1 & 20.0\% & 11 & do \\
Matt & 2 & 1 & 50.0\% & 23 &  \\
Peter & 1 & 1 & 100.0\% & 4 &  \\
Barbara & 2 & 0 & 0.0\% & 11 &  \\
Fraser & 4 & 2 & 50.0\% & 44 & is, go \\
Michelle & 1 & 1 & 100.0\% & 9 & are \\
Shem & 2 & 1 & 50.0\% & 4 & to \\
Becky & 2 & 1 & 50.0\% & 18 & have \\
Dominic & 2 & 0 & 0.0\% & 19 &  \\
Roman & 2 & 0 & 0.0\% & 2 &  \\
\hline
Average & 3 & 1 & 30.9\% & 21.65 & \\
\bottomrule
\end{tabular}
\end{table}
\FloatBarrier


\FloatBarrier
\begin{table}[!h]
\centering
\footnotesize
\caption{The number of children's error and correct `ACC + V' sequences and parents' `ACC + V' sequences}
\label{tab:kkkiiirrr}
\begin{tabular}{l|lll|lll|lll|lll}
\toprule
 & \multicolumn{3}{c|}{\textbf{`Me + V'}} & \multicolumn{3}{c|}{\textbf{`Her + V'}} & \multicolumn{3}{c|}{\textbf{`Him + V'}} & \multicolumn{3}{c}{\textbf{`Them + V'}} \\ \cline{2-13} 
 & Input & \begin{tabular}[c]{@{}l@{}}Child \\ error\end{tabular} & \begin{tabular}[c]{@{}l@{}}Child\\ correct\end{tabular} & Input & \begin{tabular}[c]{@{}l@{}}Child \\ error\end{tabular} & \begin{tabular}[c]{@{}l@{}}Child\\ correct\end{tabular} & Input & \begin{tabular}[c]{@{}l@{}}Child \\ error\end{tabular} & \begin{tabular}[c]{@{}l@{}}Child\\ correct\end{tabular} & Input & \begin{tabular}[c]{@{}l@{}}Child \\ error\end{tabular} & \begin{tabular}[c]{@{}l@{}}Child\\ correct\end{tabular} \\ \hline
Abe & 129 & 20 & 126 & 4 & 43 & 2 & 5 & 1 & 15 & 9 & 4 & 17 \\
Ben & 0 & 0 & 7 & 1 & 18 & 0 & 3 & 2 & 4 & 0 & 0 & 0 \\
Eleanor & 231 & 23 & 46 & 10 & 0 & 6 & 4 & 0 & 4 & 19 & 4 & 9 \\
Geraldine & 23 & 1 & 8 & 1 & 1 & 0 & 2 & 0 & 0 & 1 & 0 & 0 \\
Lara & 488 & 28 & 228 & 60 & 7 & 13 & 11 & 0 & 1 & 60 & 2 & 10 \\
Nathaniel & 37 & 3 & 2 & 2 & 0 & 1 & 6 & 0 & 0 & 15 & 1 & 1 \\
Ross & 22 & 15 & 143 & 0 & 15 & 4 & 11 & 22 & 9 & 3 & 1 & 17 \\
Thomas & 1195 & 25 & 198 & 175 & 5 & 8 & 103 & 2 & 2 & 423 & 17 & 7 \\
Adam & 55 & 31 & 505 & 15 & 4 & 3 & 24 & 1 & 9 & 28 & 1 & 0 \\
Carl & 34 & 6 & 14 & 5 & 0 & 0 & 8 & 0 & 0 & 0 & 1 & 0 \\
Emily & 1 & 4 & 24 & 1 & 0 & 0 & 0 & 1 & 3 & 3 & 0 & 2 \\
Jillian & 36 & 1 & 16 & 7 & 1 & 1 & 2 & 0 & 0 & 10 & 0 & 3 \\
Laura & 94 & 19 & 62 & 17 & 0 & 4 & 2 & 0 & 8 & 8 & 3 & 1 \\
Nicole & 163 & 18 & 4 & 11 & 2 & 0 & 11 & 0 & 0 & 24 & 12 & 3 \\
Ruth & 216 & 425 & 32 & 51 & 0 & 0 & 11 & 0 & 0 & 24 & 1 & 0 \\
Anne & 206 & 25 & 12 & 24 & 7 & 1 & 13 & 1 & 2 & 26 & 0 & 0 \\
Conor & 50 & 1 & 8 & 6 & 0 & 1 & 9 & 0 & 2 & 18 & 1 & 3 \\
Emma & 4 & 3 & 29 & 1 & 1 & 0 & 0 & 1 & 2 & 3 & 0 & 3 \\
Jimmy & 11 & 33 & 81 & 0 & 0 & 0 & 1 & 0 & 1 & 3 & 2 & 9 \\
Liz & 143 & 15 & 9 & 3 & 0 & 1 & 4 & 0 & 0 & 24 & 0 & 19 \\
Nina & 355 & 12 & 143 & 35 & 103 & 25 & 37 & 25 & 14 & 36 & 0 & 0 \\
Sarah & 171 & 6 & 185 & 28 & 19 & 3 & 13 & 2 & 3 & 8 & 5 & 1 \\
Tow & 92 & 0 & 5 & 28 & 9 & 1 & 8 & 1 & 0 & 8 & 0 & 0 \\
Aran & 153 & 6 & 69 & 34 & 2 & 2 & 30 & 0 & 1 & 60 & 1 & 1 \\
Courtney & 7 & 0 & 3 & 8 & 2 & 0 & 0 & 0 & 6 & 12 & 0 & 0 \\
Eve & 89 & 6 & 83 & 1 & 3 & 0 & 1 & 0 & 0 & 4 & 1 & 0 \\
Joel & 92 & 8 & 16 & 7 & 1 & 0 & 22 & 0 & 0 & 22 & 0 & 3 \\
Matt & 143 & 4 & 45 & 5 & 2 & 1 & 23 & 6 & 3 & 38 & 2 & 1 \\
Peter & 42 & 38 & 51 & 3 & 0 & 1 & 0 & 0 & 9 & 6 & 2 & 2 \\
She & 46 & 7 & 13 & 17 & 5 & 0 & 10 & 0 & 0 & 0 & 0 & 0 \\
Trevor & 0 & 6 & 25 & 0 & 0 & 2 & 0 & 2 & 0 & 0 & 0 & 0 \\
Barbara & 35 & 2 & 8 & 20 & 1 & 0 & 3 & 0 & 1 & 12 & 2 & 2 \\
David & 8 & 2 & 6 & 6 & 0 & 0 & 10 & 0 & 0 & 10 & 0 & 0 \\
Fraser & 615 & 22 & 139 & 16 & 3 & 4 & 48 & 4 & 2 & 157 & 5 & 10 \\
John & 107 & 4 & 1 & 9 & 0 & 0 & 1 & 0 & 0 & 22 & 0 & 0 \\
Michelle & 19 & 2 & 15 & 18 & 0 & 0 & 4 & 0 & 1 & 11 & 5 & 1 \\
Rachel & 8 & 0 & 5 & 5 & 10 & 3 & 0 & 1 & 0 & 6 & 0 & 0 \\
Shem & 5 & 2 & 26 & 2 & 0 & 0 & 2 & 2 & 2 & 4 & 6 & 0 \\
Warren & 42 & 8 & 12 & 8 & 0 & 0 & 12 & 0 & 5 & 21 & 1 & 2 \\
Becky & 176 & 5 & 22 & 15 & 6 & 6 & 25 & 7 & 0 & 24 & 2 & 1 \\
Dominic & 185 & 8 & 26 & 40 & 0 & 0 & 8 & 0 & 0 & 40 & 2 & 0 \\
Gail & 40 & 1 & 2 & 11 & 11 & 10 & 24 & 3 & 0 & 41 & 0 & 0 \\
Johnny & 4 & 0 & 7 & 0 & 0 & 0 & 0 & 0 & 0 & 1 & 0 & 0 \\
Naomi & 45 & 23 & 26 & 2 & 9 & 1 & 1 & 2 & 0 & 4 & 0 & 2 \\
Roman & 7 & 1 & 57 & 2 & 0 & 3 & 0 & 3 & 2 & 2 & 2 & 3 \\
Stuart & 12 & 1 & 23 & 1 & 1 & 3 & 3 & 0 & 0 & 3 & 0 & 1\\
\bottomrule
\end{tabular}
\end{table}
\FloatBarrier


\subsection{Conclusion}

This section reviewed and re-examined \cite{tomasello2000,tomasello2003}'s hypothesis that children make errors like `\textit{her} do it' by  mistakenly imitating parents' utterances like `\textit{Let her do it}'. By replicating \cite{kirjavainen2009can}'s study on four accusative pronouns (\textit{me, her, him, them}) with 46 children's data, no significant correlations were found between the proportions of parents' `\textit{me/him/them} + V' sequences and children's error `\textit{me/him/them} + V' proportions. Only a weak positive correlation (r = 0.38*) was found between parents' `\textit{her} + V' proportions and children's error `\textit{her} + V' proportions. It is unclear why the positive correlation between the proportions of parents' `\textit{me} + V' and children's `\textit{me} + V' errors in \cite{kirjavainen2009can} was not replicated, although the data used in this study is highly correlated with the data in \cite{kirjavainen2009can}. Furthermore, there's no significant strong correlation between the proportions of children's correct `ACC + V' and error `ACC + V' or parents' `ACC + V' sequences. These correlation results suggest that children's `ACC + V' errors are not related to parents' `ACC + V' sequences or their own correct `ACC + V' sequences. In addition, if parents' `ACC + V' sequences lead to children's `ACC + V' errors, the proportions of `ACC + V' sequences should be different for the children who made errors and the children who did not made errors. However, the t-test showed that the mean of parents' `ACC + V' proportions were not significantly different. Furthermore, the verbs in parents' `ACC + V' sequences and children's `ACC + V' errors were not highly overlapped. Children created `ACC + V' errors with novel verbs that they didn't hear from parents' input, suggesting that parents' `ACC + V' sequences are unlikely to be the source for children's `ACC + V' errors. 

In addition, children's `\textit{me/her/him/them} + V' productions were not correlated, while different pronouns in parents' `ACC + V' sequences were generally correlated with each other. This is an interesting contrast, suggesting that children at this age probably still treat `\textit{me', `her', `him', `them}' as distinctive individuals instead categorizing them as the accusative pronouns. 
In conclusion, there is not enough evidence for \cite{tomasello2000,tomasello2003}'s hypothesis that children made errors like `\textit{her} open it' because they mistakenly imitated parents' utterances like `Let \textit{her} open it'. There's no significant strong correlation between children's `ACC + V' errors and parents' `ACC + V' sequences. In addition, children use many novel verbs in the `ACC + V' errors which were not found in parents' `ACC + V' input, suggesting that children did not simply copied parental speech and made errors. 

