\section{Can Morphological Features Explain Pronoun Case Errors?}
\subsection{Pronoun Case Paradigm}
Apart from the syntactic structure and parents' input, the morphological structure of English pronouns could also affect children's pronoun case errors. In morphology, a paradigm is a set of related words that are different in certain grammatical features but all associated with a stem form. The paradigm can be divided into two types: the general paradigm and word specific paradigm. In general paradigm, different words follow a regular pattern of inflection. For example in Table \ref{tab:turkish}, the absolute case is the base form for Turkish pronouns and other cases are consistently derived by regular rules, such as adding -\textit{i/u} for accusative case or adding -\textit{de/da} for locative case. Comparing to Turkish, English's pronouns form a word specific paradigm. As shown in Table \ref{tab:englishparadigm}, each case in English has its own form and do not follow a regular rule. 


\FloatBarrier
\begin{table}[!h]
\centering
\caption{Turkish Pronoun Paradigm}
\label{tab:turkish}
\begin{tabular}{l|lll|lll}
\toprule
 & \multicolumn{3}{|c}{Singular} & \multicolumn{3}{|c}{Plural} \\
 \hline
Case & 1st person & 2nd person & 3rd person & 1st person & 2nd person & 3rd person \\
\hline
Absolute & ben & sen & o & biz & siz & onlar \\
Accusative & beni & seni & onu & bizi & sizi & onlari \\
Dative & bana & sana & ona & bize & size & onlara \\
Locative & bende & sende & onda & bizde & sizde & onlarda \\
Ablative & benden & senden & ondan & bizden & sizden & onlardan \\
Genitive & benim & senin & onun & bizim & sizin & onlarin\\
\bottomrule
\end{tabular}
\end{table}
\FloatBarrier

\FloatBarrier
\begin{table}[!h]
\centering
\caption{English Pronoun Paradigm and Errors}
\label{tab:englishparadigm}
\begin{tabular}{l|lll|lll}
\toprule
 & \multicolumn{3}{|c}{Singular} & \multicolumn{3}{|c}{Plural} \\ \hline
Case & 1st person & 2nd person & 3rd person & 1st person & 2nd person & 3rd person \\ \hline
Nomninative & I & you & he/she/it & we & you & they \\
Accusative & me & you & him/her/it & us & you & them \\
Genitive & my & your & his/her/its & our & your & their\\
\bottomrule
\end{tabular}
\end{table}
\FloatBarrier

\cite{rispoli1994,rispoli1998, rispoli2005} proposed that children also forms the a mental paradigm of English pronouns based on the grammatical features such as person, number, case and gender; and pronoun case errors occur when they fail to retrieve a given form. Rispoli further discussed three scenarios where children could make mistakes with mental pronoun paradigm:
\begin{enumerate}
    \item Paradigm Building Error: Children might mistakenly build a general paradigm for English pronouns instead of a word-specific one. Children would try to apply rules to get the right form of the pronoun. If the children rely on a general paradigm, rule-applying errors should occur across all pronouns. For example, the children might be misled by `\textit{you}' and `\textit{it}' and believe that the nominative and accusative cased pronouns share the same form. They are likely to extend this rule to other pronouns too, making errors such as `\textit{me} for \textit{I}' or `\textit{her} for \textit{she}'. In addition, children won't make errors like `\textit{me}' for `\textit{my}' or `\textit{him}' for `\textit{his}' since there's no rule indicating that the accusative case and the genitive case also share the same form. 
    \item Partial Paradigm Error: Children realize that English has a word specific paradigm; however they might miss some features when building the paradigm. For example, the children might only recognize nominative and accusative case and fail to identify the genitive case. In this case, the children are likely to fill the unknown genitive case cell with nominative or accusative pronouns, making errors like `\textit{I/me}' for `\textit{my}', `\textit{he/him}' for \textit{his} and `\textit{we/us}' for `\textit{our}'. 
    \item Retrieval Error: Children have the correct word specific paradigm for English pronouns. However, when they try to recall the correct form, they make retrieval errors. If pronoun case errors are retrieval errors, the errors are unlikely to be related to each other. In addition, the more various pronouns the children try to produce, the more likely they make retrieval errors. 
\end{enumerate}

\subsection{Testing Paradigm Building Error}

\cite{rispoli1994} tested if the morphological features could explain children's pronoun case error patterns; and if so, it could be the evidence that children have made Paradigm Building Error that they erroneously built a general paradigm with the misconstructed rules. He examined two types of errors: nominative overextension (e.g. `\textit{I}' for `\textit{me/my}' errors) and accusative overextension (e.g. `\textit{me}' for `\textit{I/my}' errors). He hypothesized that children establish minimal phonetic consistencies, or phonetic cores, for pronouns of different persons. For example, the phonetic core of third person singular masculine pronoun is `\textit{h-}', since it is shared in all three cases `\textit{he/him/his}'. Similarly, the phonetic core of third person plural pronoun is `\textit{th-}', since it's shared in `\textit{they/them/their}'. For first person singular pronoun and third person singular feminine pronoun, it's difficult to form phonetic cores, since there is no common element shared by `\textit{I/me/my}' and `\textit{she/her/her}'. Rispoli further made the prediction that only pronouns with shared phonetic cores will be overextended. He tested it on 12 children's longitudinal spontaneous data from 1;0 - 3;0. As shown in Table \ref{tab:rispoliandchildes}, the average nominative overextension rate of  `\textit{I}' (0.00) and `\textit{she}' (0.00) were significantly lower than `\textit{he}' (0.09) and `\textit{they}' (0.04), which confirms the prediction that pronouns with phonetic cores are more likely to be overextended. However, for the accusative overextension rate, pronouns without phonemic cores `\textit{me}' (0.10) and `\textit{her}' (0.47) have the higher overextension rate than `\textit{him}' (0.05) and `\textit{them}' (0.06), which contradicts the prediction. In addition, the 46 children's data could not confirm the predictions either. In general, the children in CHILDES have lower nominative overextension rate than the children in \cite{rispoli1994}. The average nominative and accusative overextension rates of pronouns with or without the phonetic cores do not differ for the children's data in the CHILDES. In addition, the low overextension rate indicates that children consistently use the correct form of the pronoun.  
\FloatBarrier
\begin{table}[!h]
\centering
\caption{Average overextension rate in \cite{rispoli1994}'s data and CHILDES's data}
\label{tab:rispoliandchildes}
\begin{tabular}{ll|l|l|l|l}
\toprule
 &  & \multicolumn{2}{l|}{Nominative Overextension} & \multicolumn{2}{l}{Accusative Overextension} \\ \hline
\multicolumn{1}{l|}{} & Pronoun & \cite{rispoli1994} & CHILDES & \cite{rispoli1994} & CHILDES \\ \hline
\multicolumn{1}{l|}{\multirow{2}{*}{\begin{tabular}[c]{@{}l@{}}Without \\ Phonetic Cores\end{tabular}}} & 1psg & 0.00 & 0.000 & 0.10 & 0.08 \\
\multicolumn{1}{l|}{} & 3psg fem. & 0.00 & 0.001 & 0.47 & 0.10 \\ \hline
\multicolumn{1}{l|}{\multirow{2}{*}{\begin{tabular}[c]{@{}l@{}}With \\ Phonetic Cores\end{tabular}}} & 3psg mas. & 0.09 & 0.004 & 0.05 & 0.08 \\
\multicolumn{1}{l|}{} & 3pl & 0.04 & 0.001 & 0.06 & 0.10\\
\bottomrule
\end{tabular}
\end{table}
\FloatBarrier

In addition, \cite{rispoli1998} also investigated `\textit{me}' for `\textit{I}' errors and `\textit{my}' for `\textit{I}' in children's production using the same 12 children's data to test if children's pronoun case errors could be explained by morphological rules. He hypothesized that the difference in the overextension of `\textit{me}' for `\textit{I}' and `\textit{my}' for `\textit{I}' lies in the different misconstruction of the phonetic core for first person singular pronoun. When children would overextend `\textit{me}' for `\textit{I}' because the believe that the phonetic core is`\textit{m-}', since `\textit{m-}' is shared in `\textit{me}' and `\textit{my}'. When they use `\textit{my}' for `\textit{I}', they would believe that  `\textit{-ai}' is the phonetic core since `\textit{-ai}' is shared in `\textit{I}' and `\textit{my}'. Based on this hypothesis, children should not make `\textit{me}' for `\textit{I}' and `\textit{my}' for `\textit{I}' at the same time. Rispoli counted both errors in children's production using the same 12 children's data as in \cite{rispoli1994}. He further divided children into \textit{me} children, who predominately overextend `\textit{me}', and \textit{my} children, who predominately overextend `\textit{my}'. By calculating the percentages of `\textit{me}' and `\textit{my}' of the total `\textit{I}' substitution errors, he found 9 `\textit{me}' children whose \textit{me} error percentage is over 70\%, 2 `\textit{my}' children whose \textit{my} error percentage is over 70\% and 1 balanced child whose errors consist of 56\% \textit{me} and 44\% of \textit{my}, as shown in Table \ref{memychildren}. Therefore, Rispoli concluded that children who made \textit{`me'} errors and \textit{`my'} errors are mutually exclusive, thus suggesting that children make pronoun case errors because they have a general paradigm. His findings were replicated on the 46 children's data in CHILDES. Among 46 children, 25 of them made both `\textit{me}-for-I' and `\textit{my}-for-I' errors. Using Rispoli's criterion, there are 15 `\textit{me}' children, with an average `\textit{me}-for-I' rate of 0.85; 3 `\textit{my}' children, with an average `\textit{my}-for-I' rate of 0.84; and 7 balanced children, whose average `\textit{me}-for-I' rate is 0.55 and `\textit{my}-for-I' rate is 0.45. The data in CHILDES does not confirm the Rispoli's prediction since 7 out 25 children have a balanced `\textit{me}' and `\textit{my}' errors. In addition, children also make `\textit{me}-for-my' errors and `\textit{my}-for-me' errors. If children make `\textit{me}-for-I' errors because they believe the phonetic core is `\textit{m-}', they should not make any `\textit{my}-for-me' errors. However, 2 of the `\textit{me}' children overextended `\textit{my}' for `\textit{me}'. Similarly, if children believed the phonetic core is `\textit{-ai}', they would only overextend `\textit{my}' but not `\textit{me}'. However, 2 of the `\textit{my}' children also made `\textit{me}-for-my' errors. These evidence show that the pattern of pronoun case errors does not fit a morphological rule, which suggests that children do not have a general paradigm for English pronouns.  

\FloatBarrier
\begin{table}[!h]
\caption{\textit{Me}-error rate and \textit{My}-error rate for `\textit{me} children' and `\textit{my} children' in \cite{rispoli1998} and CHILDES data}
\centering
\label{memychildren}
\begin{tabular}{l|l|l|l|l|l|l}
\toprule
 & \multicolumn{2}{l|}{\textit{me} child} & \multicolumn{2}{l|}{\textit{my} child} & \multicolumn{2}{l}{balanced} \\ \hline
 & N & \begin{tabular}[c]{@{}l@{}}Average \\ \textit{me} rate\end{tabular} & N & \begin{tabular}[c]{@{}l@{}}Average \\ \textit{my} rate\end{tabular} & N & \begin{tabular}[c]{@{}l@{}}Average\\ \textit{me} rate\end{tabular} \\ \hline
\cite{rispoli1994} & 9 & 0.83 & 2 & 0.75 & 1 & 0.56 \\ \hline
CHILDES & 15 & 0.85 & 3 & 0.84 & 7 & 0.55\\
\bottomrule
\end{tabular}
\end{table}
\FloatBarrier

Based on the replication results of \cite{rispoli1994} and \cite{rispoli1998}, there is not enough evidence to show that phonetic cores affect children's pronoun case errors, suggesting that children do not have a general paradigm for English pronouns. 

\subsection{Partial Paradigm and Double-Cell Effect}
\cite{rispoli1998patterns} also investigated the possibility of treating pronoun paradigms as individuals, which would be helpful to explain that the error rate is not uniform across different pronouns. The evidence for a partial word specific paradigm would be that previous studies have observed the error rate of `\textit{her}-for-she' errors are higher than other types of pronominal errors. The pronoun \textit{her} is special since it's the form for both genitive case and accusative case. Since `\textit{her}' takes up two `cells' in the pronoun paradigm, it is going to be overextended more than the other pronoun forms. Rispoli termed `double-cell effect' for this type of morphological paradigm impact on pronoun case error pattern. In order to test if \textit{`her}' truly has a higher rate of overextension, Rispoli counted the overextension errors of third person accusative pronouns `\textit{him}', `\textit{her}' and `\textit{them}'. If the overextension rate of `\textit{her}' is higher than `\textit{him}' and `\textit{them}', the double-cell effect can be confirmed. He used the cross-sectional spontaneous speech data of 22 children between the ages of 2;6 and 4;0. As shown in Table \ref{tab:rispoli1998}, he found 197 tokens of `\textit{she}' and 192 tokens of `\textit{her'}-for-`\textit{she}' errors, creating an error rate of 49\% (192/197+192); whereas the error rate for `\textit{him}-for`\textit{he}' is only 11\% (71/574+71) and the error rate for `\textit{them}'-for'\textit{they}' is only 12\% (30/215+30). The result showed that `\textit{her}' does have a higher overextension rate than `\textit{him}' and `\textit{them}'. In addition, this pattern is also confirmed on 211 children's cross-sectional data in CHILDES. The error rate of `\textit{her}-for-she' for the children in CHILDES is 27\%, whereas the error rates of `\textit{him}-for-he' and `\textit{them}-for-they' are 3\% and 4\% respectively.  

\FloatBarrier
\begin{table}[!h]
\centering
\caption{Summary of children's third person pronoun's overextension errors in \cite{rispoli1998patterns} and in CHILDES data}
\label{tab:rispoli1998}
\begin{tabular}{l|lllll|lllll}
\toprule
 & \multicolumn{5}{l|}{\cite{rispoli1998patterns} (n = 22)} & \multicolumn{5}{l}{CHILDES (n = 211)} \\ \hline
Pronoun & Tokens & Mean & S.D & Range & \begin{tabular}[c]{@{}l@{}}Error\\ rate\end{tabular} & Tokens & Mean & S.D & Range & \begin{tabular}[c]{@{}l@{}}Error\\ rate\end{tabular} \\ \hline
 she & 197 & 9 & 15 & 0-57 &  & 563 & 2.68 & 4.81 & 0-31 &  \\
 her for she & 192 & 9 & 10 & 0-36 & 49\% & 210 & 1 & 3.5 & 0-42 & 27\% \\ \hline
 he & 574 & 26 & 21 & 1-43 &  & 1841 & 8.77 & 16.39 & 0-196 &  \\
 him for he & 71 & 3 & 5 & 0-27 & 11\% & 63 & 0.3 & 1.48 & 0-15 & 3\% \\ \hline
they & 215 & 10 & 10 & 0-30 &  & 936 & 4.46 & 6.21 & 0-45 &  \\
 them for they & 30 & 1 & 3 & 0-11 & 12\% & 44 & 0.21 & 0.92 & 0-10 & 4\%\\
 \bottomrule
\end{tabular}
\end{table}
\FloatBarrier

Rispoli also compared the error rate of third person pronouns for each child. He selected 12 children of the 22 children who made more than 5 nominative third person pronouns, and conducted one-way ANOVA test on the error rate of `\textit{her}-for-she', `\textit{him}-for-he' and `\textit{them}-for-they'. The results of the ANOVA were significant, F = 9.9, p < 0.01, as shown in Table \ref{tab:mario}. The post-hoc t-tests showed that `hte error rate of `\textit{her}-for-she' is significantly higher than `\textit{him}-for-he' and `\textit{them}-for-they', and there is no significant difference between the error rate of `\textit{him}-for-he' and `\textit{them}-for-they'. In addition, the data in CHILDES confirmed this results. Among 211 children, 153 of them produced more than 5 third person nominative pronouns. The ANOVA results showed that the error rates for three pronouns are significantly different The post-hot t-tests also showed that the error rate of `\textit{her}-for-she' is significantly higher and erthere is no difference between `\textit{him}-for-he' and `\textit{them}-for-they'. 
\FloatBarrier
\begin{table}[!h]
\centering
\caption{The results of ANOVA and t-tests of children's error rate in \cite{rispoli1998patterns} and CHILDES}
\label{tab:mario}
\begin{tabular}{l|l|l|lll}
\toprule
 & &  & Her vs Him & Her vs Them & Him vs Them \\ \hline
 & N & ANOVA F & t-test & t-test & t-test \\ \hline
\cite{rispoli1998patterns} & 12 &  9.9** & 3.12* & 3.2** & 1.55 \\ \hline
CHILDES & 153 & 30*** & 6.68*** & 5.34*** & 1.7\\
\bottomrule
\end{tabular}
\end{table}
\FloatBarrier

The replication of \cite{rispoli1998patterns} confirms the double-cell effect, that the pronoun `\textit{her}' has a higher overextension rate than `\textit{him}' and `\textit{them}, suggesting that the morphological features could affect children's pronoun case errors. 

\subsection{Retrieval Error and the dispersion of children's pronoun production}
\label{section:rispoli20052}
Other than morphological features, \cite{rispoli2005} proposed that children's pronoun production should also affect their pronoun case error if they have a word specific paradigm for English pronouns. The more cells in the paradigm the children attempt to produce, the more likely they would make an error. To measure the dispersion of children's pronoun production, Rispoli created a measure, SDpro, which is the standard deviation of the observed distribution of pronominal attempts across the case distinct 3rd person paradigm. If the children try to produce each pronoun equally, the proportion of each pronoun would be 0.11, as in 1 out of 9 cells. However, children never produce same amount of third person pronouns. Therefore, the actual proportion is calculated as the tokens for each pronoun over the total token of all third person pronouns. The difference between the actual proportion and the equal proportion 0.11 is the deviation score of how much the actual production varies from the mean. The sum of the squared deviation scores become the variance. The SDpro is the square root of the variance. An example from \cite{rispoli2005} is used to illustrate the calculation. The actual proportion and frequency for each pronoun is shown in Table \ref{tab:SDpro} with the equal proportion 0.11 shown in the parentheses. 

\FloatBarrier
\begin{table}[!h]
\centering
\caption{Proportion of each third person pronoun across the paradigm for one child, replicated from \cite{rispoli2005}}
\label{tab:SDpro}
\begin{tabular}{l|l|lll}
\toprule
 &  & Nom & Obj & Gen \\ \hline
\multirow{2}{*}{Masc} & Proportion & 0.49 (0.11) & 0.02 (0.11) & 0.04 (0.11) \\ \cline{2-5} 
 & Frequency & 24 & 1 & 2 \\ \hline
\multirow{2}{*}{Fem} & Proportion & 0.20 (0.11) & 0.08 (0.11) & 0.14 (0.11) \\ \cline{2-5} 
 & Frequency & 10 & 4 & 7 \\ \hline
\multirow{2}{*}{Plu} & Proportion & 0.02 (0.11) & 0 (0.11) & 0 (0.11) \\
 & Frequency & 1 & 0 & 0\\
 \bottomrule
\end{tabular}
\end{table}
\FloatBarrier

For the example child, she produced total 49 third person pronouns. The proportion for each cell is calculated as the frequency of each cell divided by 49. For example, the proportion of masculine nominative 3psg pronoun is 24/49 which is 0.49. The SDpro is the sum of the squared difference between the actual proportion and the equal proportion:

\begin{aligned}
$Variance$ & =  (0.49-0.11)^2 + (0.02-0.11)^2 + (0.04-0.11)^2 + \\
& (0.20-0.11)^2 + (0.08-0.11)^2 + (0.14-0.11)^2 + \\
&  (0.02-0.11)^2 + (0-0.11)^2 + (0-0.11)^2 = 0.025\\ 
\end{aligned}

\begin{aligned}
$SDpro$ & = \sqrt{0.025} = 0.16 
\end{aligned}

Based on this calculation, the lower the SDpro is, the more diversed children's pronoun production is. Rispoli thus predicted that the SDpro should be negatively correlated with children's pronoun case error rate. He collected spontaneous speech data from 44 children ranging from 2;0 - 4;0. He tabulated the error rate of third person pronoun and age, mlu and finiteness. The detailed description of the study can be found in \ref{section:rispoli20051}. The correlation results, as in Table \ref{tab:186} showed that SDpro is negatively correlated with pronoun case error rate. However, there is no significant correlation found between SDpro with the error rate in the children's data from CHILDES. The replication results of \cite{rispoli2005} showed that there is no correlation between children's pronoun error rate and the dispersion of their pronoun production, suggesting that children make pronoun case errors regardless of how many difference pronouns they use. 

\FloatBarrier
\begin{table}[!h]
\centering
\caption{Correlation table for children's data in  \cite{rispoli1994} and in CHILDES (combination of Table \ref{tab:my80} and \ref{tab:corRis})}
\label{tab:186}
\begin{tabular}{l|llllll}
\toprule
 &  & 1 & 2 & 3 & 4 & 5 \\ \hline
\multirow{2}{*}{1.Error Rate} & Rispoli & 1 & 0.14 & 0.05 & \textbf{-0.38*} & \textbf{-0.44*} \\
 & CHILDES & 1 & 0.13 & -0.12 & -0.12 & -0.03 \\ \hline
\multirow{2}{*}{2. Age} & Rispoli &  & 1 & \textbf{0.38* }& 0.29 & \textbf{-0.34*} \\
 & CHILDES &  & 1 & \textbf{0.71***} & \textbf{0.32*} & 0.18 \\ \hline
\multirow{2}{*}{3. MLU} & Rispoli &  &  & 1 & \textbf{0.57**} & -0.30 \\
 & CHILDES &  &  & 1 & \textbf{0.62***} & -0.07 \\ \hline
\multirow{2}{*}{4. Finiteness} & Rispoli &  &  &  & 1 & \textbf{-0.33*} \\
 & CHILDES &  &  &  & 1 & -0.03 \\ \hline
\multirow{2}{*}{5. SDpro} & Rispoli &  &  &  &  & 1 \\
 & CHILDES &  &  &  &  & 1\\
 \bottomrule
\end{tabular}
\end{table}
\FloatBarrier

\subsection{Are the errors associated?}
\cite{fitzgerald2017case} investigated if there was an association between pronoun case errors. For example, if a child makes errors on the first person pronouns, is it likely that she also makes errors on the third person pronouns? If there is an association between different pronoun case errors, it could suggest that children have a underlying representation of case and the errors spread across different pronouns. Alternatively, if there is no association between pronoun case errors, which means that the co-occurrences of the errors are somewhat due to chance, the children might acquire the pronouns in an item-based fashion and features in the pronouns do not carry across the paradigm. For example, the children would use `\textit{me}' and `\textit{him}' correctly, but they are likely treat the two pronouns as individuals and fail to recognize that they are both accusative case. Therefore, association among pronoun case errors suggest that the children have a unified acquisition of case; and a non-association would suggest that children acquire individual pronoun forms in a piecemeal fashion. 
\cite{fitzgerald2017case} tested if children's third person pronoun case errors and first person pronoun case errors are associated. They included longitudinal spontaneous speech data of 43 children from the age of 1;9 to 3;0. The created a contingency table with the number of children who produced case errors on first person pronouns and third person pronouns, as shown in Table \ref{tab:lakkll}. 
\FloatBarrier
\begin{table}[!h]
\centering
\caption{Contingency table of the number of children who made errors on 1st person pronouns and 3rd person pronouns in \cite{fitzgerald2017case}}
\label{tab:lakkll}
\begin{tabular}{llll}
\toprule
 &  & \multicolumn{2}{l}{Made 3rd person errors?} \\ \hline
 &  & Yes & No \\
\multirow{2}{*}{Made 1st person errors?} & Yes & 23 & 7 \\
 & No & 5 & 8\\
 \bottomrule
\end{tabular}
\end{table}
\FloatBarrier
There are 23 people who made both 1st and 3rd person errors, 7 of them only made 1st person errors, 5 of them only made 3rd person errors and 8 of them didn't make any errors. A chi-squared test of independence showed that there is an association between first person pronouns and third person pronouns, $\chi^2$(1, N=43) = 4.27, p = 0.04. Therefore, \cite{fitzgerald2017case} concluded that children who make 1st person errors are also likely to make 3rd person errors, and the association between two persons suggest that children have a unified case system that spread across all pronoun forms in the paradigm and didn't learn pronouns in an item-based fashion. 

The chi-square of association was replicated with the 46 children's longitudinal data in CHILDES, as shown in Table \ref{tab:kkaall}. Since the three cells are less than 5, Fisher's exact test was used. The result shows that the odds ratio is 1 and p >0.05, suggesting that there is no association between the errors on first person pronouns and third person pronouns. 

\FloatBarrier
\begin{table}[!h]
\centering
\caption{Contingency table of the number of children who made errors on 1st person pronouns and 3rd person pronouns in CHILDES data}
\label{tab:kkaall}
\begin{tabular}{llll}
\toprule
 &  & \multicolumn{2}{l}{Made 3rd person errors?} \\ \hline
 &  & Yes & No \\
\multirow{2}{*}{Made 1st person errors?} & Yes & 43 & 1 \\
 & No & 2 & 0\\
 \bottomrule
\end{tabular}
\end{table}
\FloatBarrier

Although p = 0.04 in the chi-square test in \cite{fitzgerald2017case} showed that there is significant association, it should be treated with caution since p values are sensitive to sample size. Therefore, phi-coefficient was calculated to measure the strength of the association. Phi-coefficient is a number between [-1,1], which is analogously to correlation coefficient \citep{greenwood1996guide}. The calculation of phi-coefficient can be derived from chi-square
\begin{aligned}
$\phi$ = \sqrt{\display\frac{\chi^2}{n}}
\end{aligned}, where \textit{n} is the total number of observations. The phi-coefficient for the contingency table in \cite{fitzgerald2017case} is 0.31, suggesting that there is a weak association between 1st person errors and 3rd person errors. 

In addition, contingency tables were tabulated for other features of pronouns, include case (accusative case or non-accusative case), gender (masculine or feminine) and number (singular or plural) in order to test if there is association between different features. A significant association was only found in genders, but not in all other features, as shown in Tables \ref{tabbbbb: 1} - \ref{tabbbbbb2}. The phi coefficient was calculated to measure the strength of the association between case errors on feminine pronouns and masculine pronouns. The result shows that phi coefficient is 0.33, suggesting that there's only a weak association between feminine pronoun errors and masculine pronoun errors. 

\FloatBarrier
\begin{table}[!h]
\centering
\caption{Contingency table of the number of children who made errors on \textbf{Non-nominative} errors and \textbf{Other} errors}
\label{tabbbbb: 1}
\begin{tabular}{llll}
\toprule
 &  & \multicolumn{2}{l}{\begin{tabular}[c]{@{}l@{}}Made Non-nominative errors?\\ (ACC-NOM, GEN-NOM)\end{tabular}} \\ \hline
 &  & Yes & No \\
\multirow{2}{*}{\begin{tabular}[c]{@{}l@{}}Made other case errors?\\ (NOM-ACC, ACC-GEN, GEN-ACC)\end{tabular}} & Yes & 40 & 0 \\
 & No & 6 & 0\\
 \hline
  & \multicolumn{3}{l}{Fisher's Exact p >0.05}\\
\bottomrule
\end{tabular}
\end{table}
\FloatBarrier

\FloatBarrier
\begin{table}[!h]
\centering
\caption{Contingency table of the number of children who made errors on \textbf{Masculine} pronouns and \textbf{Feminine} pronouns}
\begin{tabular}{llll}
\toprule
 &  & \multicolumn{2}{l}{Made errors on masc. pronouns?} \\ \hline
 &  & Yes & No \\
\multirow{2}{*}{Made errors on fem. pronouns?} & Yes & 26 & 6 \\
 & No & 6 & 8 \\
 \hline & \multicolumn{3}{l}{\textbf{$\chi^2$ = 5.09}*, p = 0.02, $\phi$ = 0.33}\\
\bottomrule
\end{tabular}
\end{table}
\FloatBarrier

\FloatBarrier
\begin{table}[!h]
\centering
\caption{Contingency table of the number of children who made errors on \textbf{Singular} pronouns and \textbf{Plural} pronouns}
\label{tabbbbbb2}
\begin{tabular}{llll}

\toprule
 &  & \multicolumn{2}{l}{Made errors on singular pronouns?} \\ \hline
 &  & Yes & No \\
\multirow{2}{*}{Made errors on plural pronouns?} & Yes & 40 & 2 \\
 & No & 4 & 0\\
 \hline
 & \multicolumn{3}{l}{Fisher's Exact p >0.05}\\
 \bottomrule
\end{tabular}
\end{table}
\FloatBarrier

Furthermore, associations between individual pronoun errors were also investigated. For example, if the children make a `\textit{me}-to-I' error, how likely are they also make `\textit{him}-to-he' or `\textit{her}-to-she' error?  To ensure that the cells in the contingency tables are not filled with 0s, only ACC-NOM errors were counted since they are the most popular among the children. Chi-square test (or Fisher's exact if the value in any cell is smaller than 5) was used to detect association first. If there is a significant association, phi coefficient test was used to measure the strength of the association. As shown in Table \ref{table: long123}, there is no significant association between most of the pairs of accusative pronouns. The only pair of pronouns that shows a significant association is `\textit{him}-to-he' and `\textit{her}-to-she' errors. The phi coefficient is 0.37, suggesting that there is only a weak correlation. 

By replicating \cite{fitzgerald2017case}, there is significant strong association found between errors on 1st person pronouns and 3rd person pronouns. Moreover, this study also grouped pronoun errors by different features such as case, gender and number. A significant correlation was found only between gender, but not for case and number. In addition, the phi coefficient suggest that the association between feminine and masculine pronoun is weak. Furthermore, this study also tested the association between different pronouns in ACC-NOM errors. Similarly, there is no significant correlation in all the accusative pronoun pairs except for `\textit{him}' and `\textit{her}', which also has a weak correlation based on phi-coefficient test. 

\FloatBarrier
\begin{table}[!h]
\centering
\caption{Contingency table of the number of children who made errors on \textit{me,her,him,them}}
\label{table: long123}
\begin{tabular}{llll}
\toprule
Me vs Her &  & \multicolumn{2}{l}{Made me-for-I errors?} \\ \hline
 &  & Yes & No \\
\multirow{2}{*}{Made her-for-she case errors?} & Yes & 27 & 3 \\
 & No & 14 & 2 \\
 & \multicolumn{3}{l}{Fisher's Exact p \textgreater{}0.05} \\ \hline
Me vs Him &  & \multicolumn{2}{l}{Made me-for-I errors?} \\ \hline
 &  & Yes & No \\
\multirow{2}{*}{Made him-for-he case errors?} & Yes & 24 & 2 \\
 & No & 17 & 3 \\
 & \multicolumn{3}{l}{Fisher's Exact p \textgreater{}0.05} \\ \hline
Me vs Them &  & \multicolumn{2}{l}{Made me-for-I errors?} \\ \hline
 &  & Yes & No \\
\multirow{2}{*}{Made them-for-they case errors?} & Yes & 33 & 3 \\
 & No & 8 & 2 \\
 & \multicolumn{3}{l}{Fisher's Exact p \textgreater{}0.05} \\ \hline
Her vs Him &  & \multicolumn{2}{l}{Made her-for-she errors?} \\ \hline
 &  & Yes & No \\
\multirow{2}{*}{Made him-for-he case errors?} & Yes & 21 & 5 \\
 & No & 9 & 11 \\
 & \multicolumn{3}{l}{\begin{tabular}[c]{@{}l@{}} \textbf{$\chi^2$ = 6.37*}, p = 0.011\\ $\phi$ = 0.37\end{tabular}} \\ \hline
Her vs Them &  & \multicolumn{2}{l}{Made her-for-she errors?} \\ \hline
 &  & Yes & No \\
Made them-for-they case errors? & Yes & 25 & 11 \\
 & No & 5 & 5 \\
 & \multicolumn{3}{l}{$\chi^2$ = 1.30, p = 0.25} \\ \hline
Him vs Them &  & \multicolumn{2}{l}{Made him-for-he errors?} \\ \hline
 &  & Yes & No \\
\multirow{2}{*}{Made them-for-they case errors?} & Yes & 22 & 14 \\
 & No & 4 & 6 \\
 & \multicolumn{3}{l}{Fisher's Exact p \textgreater{}0.05}\\
 \bottomrule
\end{tabular}
\end{table}
\FloatBarrier
\subsection{Conclusion}
This section reviewed how morphological features affect children's pronoun case errors. The morphological explanation hypothesized that children build a paradigm for English pronouns based on features such as case, number, person and gender. This hypothesis made several predictions. First, children don't make `\textit{me}-for-I' errors and `\textit{my}-for-I' errors at the same time since they were driven by antagonistic morphological rule. However, the replication of \cite{rispoli1994} showed that 28\% of the children make about the similar proportions of both errors. Second, the hypothesis predicted that the more cells the children attempt to produce in the paradigm, the more likely they are going to make errors. However, the replication \cite{rispoli2005} showed that there is no correlation between pronoun case error rate and the dispersion of children's pronoun production. Moreover, the hypothesis predicted that pronoun case errors should be associated, since the error should affect the pronouns in the paradigm indiscriminately. However, the replication of \cite{fitzgerald2017case} showed that for most of the pronoun case errors, no significant associations were found. There's only a significant weak association found between feminine and masculine pronouns. In addition, the hypothesis also predicted that the error rate on \textit{her} should be higher than `\textit{him}' and `\textit{them}', since it takes two cells in the paradigm (accusative and genitive), which is termed as double-cell effect. The replication of \cite{rispoli1998patterns} did confirm that the error rate on \textit{her} is significantly higher than `\textit{him}' and `\textit{them}'. However, there is no other evidence suggesting that morpholgoical feature of `\textit{her}' is the reason for the curious higher error rate. The results in this section showed that there is not enough evidence to conclude that pronoun case errors follow certain morphological patterns, or children have a pronoun paradigm. 