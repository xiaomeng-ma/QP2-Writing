\section{Study 2. Can Children's Verbal Use Explain Pronoun Case Errors? }
\subsection{Nominative case licensing}
The syntactic explanations focus on the non-nominative subject errors, where the objective pronoun and the genitive pronoun are used as the subject (e.g. *\textit{me/my} eat it). In the syntactic theory, all structural cases are licensed by the head of a projection \citep[see][]{chomsky2000minimalist}. The nominative case is licensed by the \textsc{+finite} feature at the InflP level \citep[e.g.][]{haider1985unified, alec1991case} and the genitive case is licensed by the head of the DP \citep[e.g.][]{ritter1991two}. There are still debates about how accusative case is assigned but it is agreed that it is assigned at the VP level or PP level. Other than structural cases, there is a default case in English, which is a case that doesn't denote grammatical relations or theta roles, but a filler when the case system failed to be checked. In English, the default case is the accusative case \citep{schutze1997, schutze2001}. For example, in sentences like `\textit{Who wants a cookie? - \textbf{Me}}', or `\textit{\textbf{Me} too}.' the accusative cased pronoun `me' is the default. Since nominative case is licensed by the \textsc{+finite}, when a non-finite verb is used in a sentence, the system fails to check for nominative case. Thus the subject of the sentence will get the default case, which is the objective case in English. For example, in the sentence `I went to school yesterday', the verb `went' is inflected, thus the nominative case can be checked, resulting in `I' being the subject of the sentence. However, if the verb is not inflected in the sentence, the nominative case won't get checked, resulting in the default accusative case for the subject, thus the sentence would be *`\textit{me} \textit{go} to school yesterday.' Young children often experience an Optional Infinitive stage \citep{wexler1994, wexler1998, wexler2000}, during which children would use non-finite verbs in their utterances. Although the theory didn't further explain why would children knowingly omit some verbal inflections, when children omit inflections, they would end up with using the default accusative case as the subject, which leads to the non-nominative subject errors \citep{schutze1996subject}.  

The \textsc{+finite} feature in English can be realized in tense(\textsc{tns}), and/or person and number agreement (\textsc{arg}, e.g. third person singular agreement). The nominative case can be checked when any of the \textsc{+finite} features is presence on the verbs, resulting in different configurations of subject case and finite/non-finite verbs. Table \ref{tab: pattern} shows the the predicted possible configuration of subject case and finiteness. 

\FloatBarrier
\begin{table}[!h]
    \centering
    \caption{The Configurations of finitenss with subject case (adapt from \cite{pine2005testing})}
    \begin{tabular}{llll}
    \toprule
        \textsc{Finiteness} & Case  & Examples  \\
    \hline
    Predicted to occur & & \\
         \textsc{+ arg, + tns} & \textsc{+ nom} & I am going. He goes. She's gone. She went.\\
         \textsc{- arg, - tns} & \textsc{- nom} & Me going. Him go. Her gone. Her went. \\
         (Omission of the inflections)\\
         \textsc{- arg, - tns} & \textsc{+nom} & I going. He goes. She's gone. \\
       \hline
       Predicted not to occur & & \\
       \textsc{+ arg, + tns} &  \textsc{-nom} & *\textit{Me am going. *Him goes. *Her's gone.} \\
        \bottomrule
    \end{tabular}
    \label{tab: pattern}
\end{table}
\FloatBarrier
In sum, the syntactic theory predicts that when the accusative case is used as the subject in the sentence, the verb is non-finite. 

\subsection{Testing the prediction}
The prediction was first tested on three children's data in CHILDES: Nina (1;11 - 2;5), Peter (1;11 - 2;5) and Sarah (2;8 - 3;1) \citep{schutze1996subject,schutze1997}. Only the pronoun subjects occurring with the unambiguous finite verbs and the unambiguous non-finite verbs were counted. The unambiguous finite verbs include auxiliaries, modals, copulas, past tense verbs, dummy \textit{do} and main verbs with \textit{-s}; and the unambiguous non-finite verbs include null auxiliary, null copula, main verbs and auxiliaries without \textit{-s}. Therefore, utterances like \textit{I go} or \textit{me go} are ignored. 
The tokens of nominative pronouns and non-nominative pronouns with finite verbs and non-finite verbs were tabulated and reproduced in Table \ref{table:schutzetotal} and Table \ref{tab:ATOMSchutze}. 
\FloatBarrier
\begin{table}[!h]
\centering
\caption{Total subject and verb form count for Nina, Peter, Sarah}
\label{table:schutzetotal}
\begin{tabular}{lll}
\toprule
 & \multicolumn{2}{l}{Verb Form} \\ \cline{2-3} 
Subject & Finite & -Finite \\ \hline
Nominative & 726 & 261 \\
Non-Nominative & 33 & 171\\
\hline
 \multicolumn{3}{l}{$\chi^2$ = 240.80***}\\
 \multicolumn{3}{l}{Non-nominative finite rate\textsuperscript{a} = 4.34\%} \\
\multicolumn{3}{l}{Finite non-nominative rate\textsuperscript{b} = 16.17\%}\\
\bottomrule
\end{tabular}
\end{table}
\FloatBarrier
{\centering
\textsuperscript{a}Non-nominative finite rate = \displaystyle\frac{\text{Non-nominative subjects precede finite verbs}}{\text{Total pronoun subjects precede finite verbs}}\\
\textsuperscript{b}Finite non-nominative rate = \displaystyle\frac{\text{Non-nominative subjects precede finite verbs}}{\text{Total non-nominative subjects}}
\par}

1191 subject pronouns were found that precede an unambiguous finite or non-finite verb in the corpus data of Nina, Peter and Sarah. Children generally use nominative case correctly that 82.87\% of the subjects are nominative case. In total, 63.73\% of the subjects precede a finite verb and only 4.34\% of these subjects are non-nominative pronouns. For each children, the non-nominative pronoun rate varies. For first person singular pronoun, the rate is as low as 1.48\% for Peter and 3.08\% for Nina. For third person singular pronouns, it was 7.09\% for Nina and 15.38\% for Sarah. In addition, the chi-square contingency test shows that In addition, non-nominative pronouns are more likely to appear with non-finite verbs ($\chi^2$ = 240.80, p < .001), which is true for Nina and Peter, except for Sarah's third person singular pronoun. \cite{schutze2001} further suggested that the non-nominative finite rate is so low that they can be regarded as noise in the data.

However, the chi-square test result only demonstrates that non-nominative pronouns are more likely to occur with non-finite verbs; it doesn't imply that the non-nominative pronouns rarely occur with finite verbs. In addition, the non-nominative finite rate might not be a good indicator to measure how often the non-nominative pronouns precede a finite verb. Since the denominator is the total pronoun subjects pronouns precede finite verbs, if the children predominantly use nominative pronouns as the subject, the non-nominative finite rate will naturally be low. Therefore, the total number of non-nominative subjects should also be considered to see how often the non-nominative subject precede a finite verb. Among all the 204 non-nominative subjects, 16.17\% of these subjects precede a finite verb. The finite non-nominative rate ranges from 12.33\% to 33.33\% for first person singular and third person singular pronouns among different children. When non-nominative pronouns are used as subjects, the majority of them precede a non-finite verb; but it's not rare for them to appear before a finite verb. 

\FloatBarrier
\begin{table}[!h]
    \caption{Reproducing \cite{schutze1997}'s data on Nina, Peter and Sarah}
    \label{tab:ATOMSchutze}
   \begin{minipage}[t]{0.5\textwidth}
    \centering
    \subcaption{Nina 1sg Finiteness VS Case}
    \small
    \begin{tabular}{@{}lll@{}}
        \toprule
         & \multicolumn{2}{c}{Verb form}\\
         \cline{2-3}
        Subject & Finite & Non-finite \\
        \midrule
        I & 189 & 63  \\
        me + my  & 6 & 18 \\
        \hline
        \multicolumn{3}{l}{$\chi^2$ = 26.42***} \\
         \multicolumn{3}{l}{Non-nominative finite rate = 3.08\%} \\
         \multicolumn{3}{l}{Finite non-nominative rate = 25\%}\\
        \bottomrule
    \end{tabular}
\end{minipage}
\vspace{1ex}
\begin{minipage}[t]{0.5\textwidth}
    \centering
    \subcaption{Nina 3sg Finiteness VS Case}
    \small
    \begin{tabular}{@{}lll@{}}
        \toprule
         & \multicolumn{2}{c}{Verb form}\\
         \cline{2-3}
        Subject & Finite & -Finite \\
        \midrule
        he + she & 249 & 149 \\
        him + her & 19 & 135 \\
        \hline
        \multicolumn{3}{l}{$\chi^2$ = 112.13***}\\
         \multicolumn{3}{l}{Non-nominative finite rate = 7.09\%} \\\multicolumn{3}{l}{Finite non-nominative rate = 12.33\%}\\
    \bottomrule
    \end{tabular}
    \end{minipage}
\vspace{1ex}
    %\caption{Nina Distribution by Verb Type}
    \begin{minipage}[t]{0.5\textwidth}
    \centering
    \subcaption{Nina 1sg Distribution by Verb Type}
    \small
    \begin{tabular}{lllll}
    \toprule
 &  & \multicolumn{3}{l}{Subject form} \\ \cline{3-5} 
Finiteness & Verb form & I & me & my \\ \hline
\multirow{2}{*}{\textsc{+arg}} & Auxiliary & 11 & 0 & 0 \\
 & Copula & 4 & 0 & 0 \\ \hline
\multirow{3}{*}{\textsc{+tns}} & Modal & 72 & 0 & 0 \\
 & Past Tense & 47 & 1 & 5 \\
 & Dummy \textit{do} & 55 & 0 & 0 \\ \hline
\multirow{2}{*}{\textsc{-finite}} & Null aux & 52 & 2 & 10 \\
 & Null cop & 11 & 2 & 4\\
\bottomrule
\end{tabular}
\end{minipage}
\vspace{1ex}
\begin{minipage}[t]{0.5\textwidth}
    \centering
    \subcaption{Nina 3sg Distribution by Verb Type}
    \small
\begin{tabular}{llllll}
\toprule
 &  & \multicolumn{4}{l}{Subject form} \\ \cline{3-6} 
Finiteness & Verb form & he & him & she & her \\ \hline
\multirow{3}{*}{\textsc{+arg}} & Main verb + \textit{`-s'} & 6 & 0 & 1 & 0 \\
 & Auxilliary + \textit{`-s'} & 113 & 0 & 3 & 3 \\
 & Copula + \textit{`-s'} & 88 & 3 & 2 & 4 \\ \hline
\multirow{2}{*}{\textsc{+tns}} & Modal & 13 & 1 & 1 & 2 \\
 & Past Tense & 20 & 0 & 2 & 6 \\ \hline
\multirow{4}{*}{\textsc{-finite}} & Main verb - \textit{`-s'} & 91 & 8 & 5 & 60 \\
 & Auxiliary - \textit{`-s'} & 19 & 0 & 1 & 14 \\
 & Null auxiliary & 24 & 1 & 0 & 27 \\
 & Null copula & 9 & 0 & 0 & 25\\
 \bottomrule
\end{tabular}
\end{minipage}
\vspace{1ex}
   \begin{minipage}[t]{0.5\textwidth}
    \centering
    \subcaption{Peter 1sg Finiteness VS Case}
    \small
    \begin{tabular}{@{}lll@{}}
        \toprule
         & \multicolumn{2}{c}{Verb form}\\
         \cline{2-3}
        Subject & Finite & -Finite \\
        \midrule
        I & 266 & 27 \\
        me + my  & 4 & 8\\
        \hline
        \multicolumn{3}{l}{$\chi^2$ = 37.46***} \\
         \multicolumn{3}{l}{Non-nominative finite rate = 1.48\%} \\
         \multicolumn{3}{l}{Finite non-nominative rate = 33.33\%}\\
        \bottomrule
    \end{tabular}
\end{minipage}
\vspace{1ex}
   \begin{minipage}[t]{0.5\textwidth}
    \centering
    \subcaption{Peter 1sg Distribution by Verb Type}
    \small
    \begin{tabular}{lllll}
    \toprule
 &  & \multicolumn{3}{l}{Subject form} \\ \cline{3-5} 
Finiteness & Verb form & I & me & my \\ \hline
\multirow{2}{*}{\textsc{+arg}} & Auxiliary & 126 & 0 & 0 \\
 & Copula & 10 & 0 & 0 \\ \hline
\multirow{3}{*}{\textsc{+tns}} & Modal & 54 & 0 & 0 \\
 & Past Tense & 74 & 3 & 1 \\
 & Dummy \textit{do} & 2 & 0 & 0 \\ \hline
\multirow{2}{*}{\textsc{-finite}} & Null aux & 27 & 4 & 2 \\
 & Null cop & 0 & 1 & 1\\
\bottomrule
\end{tabular}
\end{minipage}
\textbf{\linebreak}
\begin{minipage}[t]{0.5\textwidth}
    \centering
    \subcaption{Sarah 3sg Finiteness VS Case}
    \small
    \begin{tabular}{@{}lll@{}}
        \toprule
         & \multicolumn{2}{c}{Verb form}\\
         \cline{2-3}
        Subject & Finite & -Finite \\
        \midrule
        she & 22  & 22  \\
        her  & 4  & 10\\
        \hline
        \multicolumn{3}{l}{$\chi^2$ = 1.97, p = 0 .16} \\
 \multicolumn{3}{l}{Non-nominative finite rate = 15.38\%} \\
\multicolumn{3}{l}{Finite non-nominative rate = 28.58\%}\\
        \bottomrule
    \end{tabular}
    \end{minipage}
\begin{minipage}[t]{0.5\textwidth}
    \centering
    \subcaption{Sarah 3sg Distribution by Verb Type}
    \small
 \begin{tabular}{llll}
\toprule
 &  & \multicolumn{2}{l}{Subject form} \\ \cline{3-4} 
Finiteness & Verb form & she & her \\ \hline
\multirow{3}{*}{\textsc{+arg}} & Main verb + \textit{`-s'} & 3 & 0 \\
 & Auxilliary + \textit{`-s'} & 6 & 0 \\
 & Copula + + \textit{`-s'} & 2 & 0 \\ \hline
\multirow{2}{*}{\textsc{+tns}} & Modal & 3 & 0 \\
 & Past Tense & 8 & 4 \\ \hline
\multirow{4}{*}{-finite} & Main verb - \textit{`-s'} & 9 & 7 \\
 & Auxiliary - \textit{`-s'} & 1 & 0 \\
 & Null auxiliary & 8 & 3 \\
 & Null copula & 4 & 0\\
 \bottomrule
\end{tabular}
\end{minipage}
\end{table}
\FloatBarrier

\cite{pine2005testing} proposed a straightforward way to test if  non-nominative preceding a finite verb is rare: by setting an upper limit as 10\%, if the non-nominative finite rate is significantly over\footnote{Binomial test was used to compare the observed non-nominative rate and 10\%.} 10\%, then it is too frequent to be considered as noise. In their testing, they used a different criteria to select verb forms. Instead of using unambiguous finite and non-finite verbs, they used agreeing verbs and non-agreeing verbs. All the \textsc{+arg} verbs are considered as agreeing verbs, and the rest of the verbs are non-agreeing verbs, including \textsc{+tns} verbs, unambiguous non-finite verbs and ambiguous verbs. They first retested Nina's, Peter's and Sarah's data on first person singular pronouns and third person singular pronouns. The non-nominative finite rate is not significant greater than 10\% except for third person singular feminine pronoun (\textit{she, her}), with a rate of 53.8\%. The re-test result for Nina's third person singular pronoun is reproduced in Table \ref{table:nina}.
\FloatBarrier
\begin{table}[!h]
\centering
\caption{Nina's 3sg feminine agreement vs case in \cite{pine2005testing}}
\label{table:nina}
\begin{tabular}{lll}
\toprule
 & She & Her \\
 \hline
\textsc{+arg} & 6 & 7  \\
Non-agreeing & 9 & 134  \\
\hline
\multicolumn{3}{l}{Non-nominative finite rate: 53.8\%*}\\
\bottomrule
\end{tabular}
\end{table}
\FloatBarrier

In addition, they did the same test on four more children, Anne, Becky and Gail, from the Manchester corpus \citep{theakston2001} and Abe from  \cite{kuczaj1977acquisition}, who produced enough non-nominative third person singular subjects with agreeing verbs that can be properly tested. The total third person singular non-nominative rate is not significantly greater than 10\% for all four children. However, the third person feminine pronoun consistently has a higher than 10\% non-nominative rate. The results are reproduced in Table \ref{tab: Pine}. 
\FloatBarrier
\begin{table}[!h]
\centering
\caption{Non-nominative rate of total subject pronouns and 3sg feminine pronoun for Anne, Becky and Gail in \cite{pine2005testing}}
\label{tab: Pine}
\begin{tabular}{llll|ll}
\toprule
 & 3psg subjects & He + She & Him, His + Her & She & Her \\ \hline
\multirow{3}{*}{Anne} & \textsc{+arg} & 141 & 5 & 8 & 4 \\
 & Non-agreeing & 73 & 6 & 11 & 3 \\ \cline{2-6} 
 & \multicolumn{3}{l|}{Non-nominative rate: 3.4\%} & \multicolumn{2}{l}{Non-nominative rate: 33.3\%*} \\ \hline
\multirow{3}{*}{Becky} & \textsc{+arg} & 239 & 16 & 26 & 13 \\
 & Non-agreeing & 80 & 2 & 22 & 0 \\ \cline{2-6} 
 & \multicolumn{3}{l|}{Non-nominative rate: 6.3\%} & \multicolumn{2}{l}{Non-nominative rate: 33.3\%*} \\ \hline
\multirow{3}{*}{Gail} & \textsc{+arg} & 146 & 13 & 14 & 9 \\
 & Non-agreeing & 48 & 16 & 3 & 10 \\ \cline{2-6} 
 & \multicolumn{3}{l|}{Non-nominative rate: 8.2\%} & \multicolumn{2}{l}{Non-nominative rate: 39.1\%*}\\
 \hline
\multirow{3}{*}{Abe} & \textsc{+arg} &113  & 14 & 5 & 14 \\
 & Non-agreeing & 104 & 35 & 5 & 34 \\ \cline{2-6} 
 & \multicolumn{3}{l|}{Non-nominative rate: 11.0\%} & \multicolumn{2}{l}{Non-nominative rate: 73.7\%*}\\
 \bottomrule
\end{tabular}
\end{table}
\FloatBarrier
Thus, \cite{pine2005testing} concluded that the data doesn't support the syntactic explanation that the non-nominative subject is the result of the use of non-finite verbs. In addition, the low non-nominative error rate in general made this prediction particularly difficult to test. 

\subsection{Re-testing the non-nominative rate}
The syntactic explanation claims that the lack of finiteness feature is the reason for non-nominative subjects. It predicts that the non-nominative pronoun almost never appears with a finite verb. However, the children's data only confirms an association between the finite verb and nominative case; it doesn't imply that the non-nominative case rarely precedes a finite verb. The average non-nominative finite rate of first person and third person singular pronoun is consistently lower than 10\%, which some argues that could be regarded as noise in the data. However, the non-nominative finite rate varies among different person and gender pronouns. When a third person singular feminine pronoun (\textit{she, her}) precedes a finite verb, more than 10\% of the pronoun is the non-nominative case (\textit{her}), which contradicts the prediction. This section reviews the testing methods and 

In \cite{schutze1996subject} and \cite{pine2005testing}, the non-nominative finite rate ($r\%$) is calculated as the number of non-nominative subjects precede a finite verb over the total subject pronouns that precede a finite verb.
\begin{equation}
\label{equ:1}
\centering
     \text{r\%} = \displaystyle\frac{\text{non-nominative finite subjects }}{\text{total finite pronoun subjects}}
\end{equation}
This method is problematic because the result is inherently small and doesn't truly reflect how often the non-nominative subjects precede a finite verb. Assume $n\%$ and $q\%$ are the percentage of the non-nominative subjects and total pronoun subjects that precede a finite verb. Equation (\ref{equ:1}) can be re-written as: 
\begin{align}
    \label{equ:2}
    \centering
   \text{r\%} & =  \displaystyle\frac{\text{non-nominative subjects $\times$ n\%}}{\text{total pronoun subjects $\times$ q\%}} =  \displaystyle\frac{\text{non-nominative subjects}}{\text{total pronoun subjects}} \times \displaystyle\frac{\text{n}}{\text{q}}\% = \text{m\%} \times \displaystyle\frac{\text{n}}{\text{q}}
\end{align}

The first part in Equation (\ref{equ:2}) is the non-nominative subject rate ($m\%$), which is inherently small. As shown in Table (\ref{table:errordata}) in the previous section, children rarely make mistakes on nominative subjects. The average nominative subject rate for first person pronoun is 98.29\%, and 99.08\% and 92.32\% for third person masculine and feminine pronoun respectively, which means that non-nominative subject rate $m\%$ ranges from 0.92\% to 7.68\%. In addition, the second part of the Equation (\ref{equ:2}) $\displaystyle\frac{\text{n}}{\text{q}}$, is always smaller than 1. Since there is a strong association between the nominative pronouns and the finite verbs, the average percentage of a subject pronoun precedes a finite verb ($q\%$) is always higher than the percentage of a non-nominative subject precedes a finite verb ($n\%$),  $q > n$. Therefore, the non-nominative rate $r\%$ calculated by this formula is always very small, which could have been mistakenly treated as the noise in the data. In addition, it is not meaningful to compare $r\%$ to an arbitrary number, such as 10\% and then conclude that if it is less than 10\% it is low enough to be treated as noise. The surface value of this formula doesn't fairly represent the true proportion of non-nominative subjects appear before a finite verb. A more careful and sensible comparison method should be established to decide if the non-nominative subjects rarely appear with a finite verb. 

The new comparison should take the non-nominative subjects rate ($m\%$) into consideration. One reason for the low non-nominative finite rate ($r\%$) is that the non-nominative subjects rate ($m\%$) is low. Instead of comparing $r\%$ to an arbitrary number to judge if $r\%$ is small enough, $r\%$ should be compared to $m\%$. If the non-nominative finite rate is smaller than the average finite rate (n\% < q\%), then r\% < m\%. If the non-nominative finite rate is equal to or greater than the average finite rate (n\% $\geq$ q\%), then r\% $\geq$ m\%. For example, if m\% = 8\%, which means that 8\% of the subjects are non-nominative pronouns, and q\% = 30\%, meaning that 30\% of the pronoun subject appears before a non-finite verb: if only 15\% of the non-nominative subjects precede a finite verb, n\% = 15\%, then $\text{r\%} = \text{m\%} \times \displaystyle\frac{\text{n\%}}{\text{q\%}} = 4\%$. However, if n\% = 36\%, meaning 36\% of the non-nominative subjects precede a finite verb, then r\% = 9.6\%. Even r\% is smaller than 10\%, it's larger than m\%, suggesting that the non-nominative finite rate is not small at all. The log-likelihood ratio test (the \textit{G}^2 test) was used to compare the non-nominatives finite rate ($r\%$) and the non-nominative subject rate ($m\%$). The G-statistic can be calculated as the natural log of the observed number (\textit{O}) divided the expected number (\textit{E}) and multiply by 2.   The formula for G is:
\begin{equation}
    \centering
    G = 2 \times \displaystyle\sum_{i = 1}^{n}(O_i \times ln(\displaystyle\frac{\text{$O_i$}}{\text{$E_i$}}))
\end{equation}
The observed numbers are the tokens of nominative finite subjects and non-nominative finite subjects. The expected number of non-nominative finite subjects is the total finite pronoun subjects times $m\%$. The expected number of nominative finite subjects is the total finite pronoun subjects times $1-m\%$. If r\% is significantly smaller than m\%, then it is reasonable to conclude that the non-nominative pronouns rarely precede a finite verb when being used as a subject. 

The data reported in \cite{schutze1997} and \cite{pine2005testing} were used to calculated the G-statistics of first person singular pronouns and third person singular pronouns. The test result of first person singular pronoun is shown in Table \ref{Tab:gtest1}. Both Nina's r\% (3.07\%) and Peter's r\% (1.48\%) are significantly smaller than their m\%, suggesting that the non-nominative first person pronoun (\textit{me} and \textit{my}) rarely precedes an infinite verb. 

\FloatBarrier
\begin{table}[!h]
\centering
\caption{G-test for 1sg pronoun for Nina and Peter (data reproduced from \cite{schutze1997})}
\label{Tab:gtest1}
\begin{tabular}{lllll}
\toprule
 &  & \begin{tabular}[c]{@{}l@{}}Total Pronoun \\ Subjects\end{tabular} & \begin{tabular}[c]{@{}l@{}}Observed \\ Finite Subjects\end{tabular} & \begin{tabular}[c]{@{}l@{}}Expected \\ Finite Subjects\end{tabular} \\ \hline
\multirow{3}{*}{Nina} & I & 903 & 189 & 169.31 \\ \cline{2-5} 
 & me + my & 137 & 6 & 25.68 \\ \cline{2-5} 
 & & \multicolumn{3}{l}{m\% = 13.17\%, r\% = 3.07\%,  \textbf{G = 24.13***}, p <0.001} \\ \bottomrule
\multirow{3}{*}{Peter} & I & 700 & 266 & 243.56 \\ 
\cline{2-5}
 & me + my & 76 & 4 & 26.44 \\ \cline{2-5} 
 & & \multicolumn{3}{l}{m\% = 9.79\%, r\% =1.48\%, \textbf{G = 31.78***}, p<0.001} \\
 \bottomrule
\end{tabular}
\end{table}
\FloatBarrier

In addition, seven more children's first pronoun subject data were added to the G-test. The selection criteria include: (1) the children have to produce enough non-nominative subject errors, which means m\% $\geq$ 2\%; (2) the child have to produce more than 2 non-nominative subjects that precede a finite verb. Only seven more children fit this selection criteria: Ruth, Nicole, Warren and Anne from the Manchester corpus \citep{theakston2001}, Laura from \cite{braunwald1971mother}, Eve from Brown corpus \citep{brown1973first} and Naomi from \cite{sachs1983talking}. The G-test results for the seven more children is shown Table \ref{tab:7more}. Only three children's r\% is significantly smaller than m\%: Ruth's 48.68\% < 60.04\%; Warren's 2.62\% < 6.56\% and Anne's 1.42\% < 3.19\%. For the other four children, their r\% is not significantly smaller than m\%, indicating that it is not rare for to have non-nominative pronouns (\textit{me} and \textit{my}) to precede a finite verb when used as the subject.  

\FloatBarrier
\begin{table}[!h]
\centering
\caption{G-test of 1sg pronoun for seven more children}
\label{tab:7more}
\begin{tabular}{lllll}
\toprule
 &  & \begin{tabular}[c]{@{}l@{}}Total Pronoun\\ Subjects\end{tabular} & \begin{tabular}[c]{@{}l@{}}Observed\\ Finite Subjects\end{tabular} & \begin{tabular}[c]{@{}l@{}}Expected\\ Finite Subjects\end{tabular} \\ \hline
\multirow{3}{*}{Ruth} & I & 591 & 78 & 60.74 \\ \cline{2-5} 
 & me + my & 888 & 74 & 91.26 \\ \cline{2-5} 
 &  & \multicolumn{3}{l}{m\% = 60.04\%, r\% = 48.68\%, \textbf{G = 7.99**}, p \textless 0.01} \\ \hline
\multirow{3}{*}{Nicole} & I & 478 & 131 & 128.73 \\ \cline{2-5} 
 & me + my & 27 & 5 & 7.27 \\ \cline{2-5} 
 &  & \multicolumn{3}{l}{m\% = 5.35\%, r\% = 3.68\%, G = 0.84, p = 0.36} \\ \hline
\multirow{3}{*}{Warren} & I & 1354 & 297 & 285.00 \\ \cline{2-5} 
 & me + my & 95 & 8 & 20.00 \\ \cline{2-5} 
 &  & \multicolumn{3}{l}{m\% = 6.56\%, r\% = 2.62\%, \textbf{G = 9.83**}, p \textless 0.01} \\ \hline
\multirow{3}{*}{Anne} & I & 1000 & 346 & 339.79 \\ \cline{2-5} 
 & me + my & 33 & 5 & 11.21 \\ \cline{2-5} 
 &  & \multicolumn{3}{l}{m\% = 3.19\%, r\% = 1.42\%, \textbf{G = 4.46*}, p \textless 0.05} \\ \hline
\multirow{3}{*}{Laura} & I & 3210 & 1188 & 1186.03 \\ \cline{2-5} 
 & me + my & 73 & 25 & 26.97 \\ \cline{2-5} 
 &  & \multicolumn{3}{l}{m\% = 2.22\%, r\% = 2.06\%, G = 0.15, p = 0.70} \\ \hline
\multirow{3}{*}{Eve} & I & 1256 & 251 & 250.81 \\ \cline{2-5} 
 & me + my & 26 & 5 & 5.19 \\ \cline{2-5} 
 &  & \multicolumn{3}{l}{m\% = 2.03\%, r\% = 1.95\%, G = 0.01, p = 0.93} \\ \hline
\multirow{3}{*}{Naomi} & I & 1820 & 657 & 651.05 \\ \cline{2-5} 
 & me + my & 39 & 8 & 13.95 \\ \cline{2-5} 
 &  & \multicolumn{3}{l}{m\% = 2.10\%, r\% = 1.20\%, G = 3.06, p =0.08}\\
 \bottomrule
\end{tabular}
\end{table}
\FloatBarrier


The results of third person singular pronouns are shown in Tables \ref{tab:himhigtest} and \ref{tab:hershegtest}. Since the third person pronoun data reproduced from \cite{pine2005testing} were tabulated using `agreeing verb' instead of `finite verb', the G-test also uses `agreeing verb' but not `finite verb'.  Although r\% is extremely low for all four children, ranging from 2.94\% to 0.75\%, none of the r\% is significantly smaller than m\$, indicating that it is not rare for the third person singular masculine non-nominative pronoun (\textit{him} and \textit{him}) to precede an agreeing verb. 

In contrast, the r\% for third person singular feminine pronouns are all relatively high for all five children, ranging from 33.33\% to 73.68\%. For Anne and Becky, their r\% (33.33\% for both) are even larger than their m\% (26.92\% and 21.31\% respectively). However, none of these r\% is significantly different from m\%, except for Nina. Nina mostly used \textit{her} as a subject that she has the highest non-nominative subject rate with m\% = 90.38\%. Her non-nominative finite rate r\% is 62.5\%, which is significantly smaller than m\%. The G-test results demonstrate that \textit{her} as a subject often precedes an agreeing verb. 
The G-test results showed that it is not rare for non-nominative third person singular pronoun (\textit{him}, \textit{his} and \textit{her}) to precede a finite/agreeing verb. The probability of a non-nominative first person pronoun (\textit{me} and \textit{my}) appearing before a finite/agreeing verb is so low that they could be treated as noise in the data. 
\FloatBarrier
\begin{table}[!h]
\centering
\caption{G-test for 3sg masculine pronoun for Nina, Anne, Becky and Gail (data reproduced from \cite{pine2005testing})}
\label{tab:himhigtest}

\begin{tabular}{lllll}
\toprule
 &  & \begin{tabular}[c]{@{}l@{}}Total Pronoun\\ Subjects\end{tabular} & \begin{tabular}[c]{@{}l@{}}Observed\\ Agreeing Subjects\end{tabular} & \begin{tabular}[c]{@{}l@{}}Expected\\ Agreeing Subjects\end{tabular} \\ \hline
\multirow{3}{*}{Nina} & he & 391 & 240 & 236.15 \\ \cline{2-5} 
 & him + his & 13 & 4 & 7.85 \\ \cline{2-5} 
 &  & \multicolumn{3}{l}{m\% = 3.22\%, r\% = 1.64\%, G = 2.37, p = 0.12} \\ \hline
\multirow{3}{*}{Anne} & he & 195 & 133 & 131.31 \\ \cline{2-5} 
 & him + his & 4 & 1 & 2.69 \\ \cline{2-5} 
 &  & \multicolumn{3}{l}{m\% = 2.01\%, r\% = 0.75\%, G = 1.43, p = 0.23} \\ \hline
\multirow{3}{*}{Becky} & he & 271 & 213 & 212.09 \\ \cline{2-5} 
 & him + his & 5 & 3 & 3.91 \\ \cline{2-5} 
 &  & \multicolumn{3}{l}{m\% = 1.81\%, r\% = 1.39\%, G = 0.24, p = 0.63} \\ \hline
\multirow{3}{*}{Gail} & he & 177 & 132 & 128.73 \\ \cline{2-5} 
 & him + his & 10 & 4 & 7.27 \\ \cline{2-5} 
 &  & \multicolumn{3}{l}{m\% = 5.35\%, r = 2.94\%, G = 1.85, p = 0.17}\\
 \bottomrule
\end{tabular}
\end{table}
\FloatBarrier

\FloatBarrier
\begin{table}[!h]
\centering
\caption{G-test for 3sg feminine pronoun for Nina, Anne, Becky and Gail (data reproduced from \cite{pine2005testing})}
\label{tab:hershegtest}
\begin{tabular}{lllll}
\toprule
 &  & \begin{tabular}[c]{@{}l@{}}Total Pronoun\\ Subjects\end{tabular} & \begin{tabular}[c]{@{}l@{}}Observed\\ Agreeing Subjects\end{tabular} & \begin{tabular}[c]{@{}l@{}}Expected\\ Agreeing Subjects\end{tabular} \\ \hline
\multirow{3}{*}{Nina} & she & 15 & 9 & 2.31 \\ \cline{2-5} 
 & her & 141 & 15 & 21.69 \\ \cline{2-5} 
 &  & \multicolumn{3}{l}{m\% = 90.38\%, r\% = 62.5\%, G = \textbf{13.43***}, p \textless{}0.001} \\ \hline
\multirow{3}{*}{Anne} & she & 19 & 8 & 8.77 \\ \cline{2-5} 
 & her & 7 & 4 & 3.23 \\ \cline{2-5} 
 &  & \multicolumn{3}{l}{m\% = 26.92\%, r\% = 33.33\%, G = 0.24, p = 0.62} \\ \hline
\multirow{3}{*}{Becky} & she & 48 & 26 & 30.69 \\ \cline{2-5} 
 & her & 13 & 13 & 8.31 \\ \cline{2-5} 
 &  & \multicolumn{3}{l}{m\% = 21.31\%, r\% = 33.33\%, G = 3.01, p = 0.08} \\ \hline
\multirow{3}{*}{Gail} & she & 17 & 14 & 10.86 \\ \cline{2-5} 
 & her & 19 & 9 & 12.14 \\ \cline{2-5} 
 &  & \multicolumn{3}{l}{m\% = 52.78\%, r = 39.13\%, G = 1.72, p = 0.19}\\
 \hline
\multirow{3}{*}{Abe} & she & 10 & 15 & 3.28 \\ \cline{2-5} 
 & her & 48 & 14 & 15.72 \\ \cline{2-5} 
 &  & \multicolumn{3}{l}{m\% = 82.76\%, r = 73.68\%, G = 0.98, p = 0.32}
\\ \bottomrule
\end{tabular}
\end{table}
\FloatBarrier



\subsection{Conclusion}
This section reviewed the syntactic explanation for non-nominative subject errors. The syntactic explanation claims that the use of the non-finite verbs leads to the non-nominative subjects in children's speech. The most important prediction of this explanation is that the non-nominative subjects almost never precede a finite verb. In general, it is very difficult to find instances where a non-nominative subject precedes a finite verb. First, children mostly use the nominative pronouns as the subject, which made the non-nominative subject uncommon. Second, children often omit verbal inflections in their speech regardless of the subject, making it difficult to find unambiguous finite verbs. The inherent scarcity of finite verbs and non-nominative subjects made the syntactic prediction very difficult to test. 

The chi-square test showed that there is an association between nominative subjects and finite verbs. Children are more likely to use finite verbs when they use nominative pronouns as the subject. In addition, the proportion of non-nominative subjects in all pronoun subjects that precede a finite verb often is a small percentage (less than 10\%), leading to some researchers to conclude that the proportion is too small that it can been seen as noise in the data. However, the small percentage of non-nominative finite verb could be the result of low rate of non-nominative subjects. For example, if only 8\% of the subjects are non-nominative, it is proportional to only have 8\% of the non-nominative subjects among all pronoun subjects that precede a finite verb, even though 8\% is a small percentage.  

To better test if it is rare for the  non-nominative subjects to appear before a finite verb, likelihood ratio test (G-test) was used to compare the non-nominative subject rate (the proportion of non-nominative subjects of all pronoun subjects) and the non-nominative finite rate (the proportion of non-nominative subjects of all the pronoun subjects that precede a finite verb). If the non-nominative finite rate is significantly smaller than the non-nominative rate, it can be concluded that it is rare for the non-nominative subjects to precede a finite verb. If there is no significant difference, the non-nominative finite rate is proportional to the non-nominative rate, which means it's not uncommon for a non-nominative subject to precede a verb. For the first person singular pronoun, five children's non-nominative finite rate is significantly smaller than non-nominative rate, suggesting that the non-nominative subjects (\textit{me} and \textit{my}) rarely precede a finite verb. For the other four children, there is no significant difference between the non-nominative rate and the non-nominative finite rate, meaning that the percentage of the non-nominative subjects (\textit{me}, \textit{my}) preceding a finite verb is proportional to the non-nominative rate. For third person singular masculine pronoun, all four children's non-nominative finite rate is not different from non-nominative rate, suggesting that it is not rare for \textit{him} and \textit{his} to precede a finite verb. For third person singular feminine pronoun, only one children's non-nominative finite rate is significantly smaller than the non-nominative rate, while the other four children showed no difference. The likelihood ratio test on first person and third person singular pronouns showed that it is not rare for a non-nominative pronoun to precede a finite verb, which is against the prediction of syntactic explanation.


In conclusion, the syntactic explanation is not sufficient enough to account for non-nominative subject errors. There is an association between finite verbs and nominative subjects; however, it is not rare for the non-nominative subjects to appear with a finite verb. The reason for the non-nominative pronouns used as the subjects can not be the use of non-finite verbs in the sentence.