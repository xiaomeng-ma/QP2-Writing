\section{Introduction}
English-speaking children make pronoun case errors from the age of two to four. Some common types of errors include: (1) using an accusative pronoun (e.g. \textit{me, him, us}) or a genitive pronoun (e.g. \textit{my, his, our}) as the subject of the sentence, see (\ref{1}) and (\ref{3}); (2) using a nominative pronoun (e.g. \textit{I, she, he}) or a genitive pronoun as the object in a sentence, see (\ref{2}) and (\ref{4}), and (3) using an accusative pronoun as the determiner, see (\ref{5}) and (\ref{6}). Researchers have categorized such errors as systematic, characteristic and frequent \citep[e.g.][]{huxley1970development, budwig1989linguistic, pelham2011input,fitzgerald2017case}.

\begin{exe}
\ex \label{01} non-nominative subject:
\begin{xlist}
\ex \label{1} {Where does \textit{him} go? - \textit{him} for \textit{he}, Becky at the age of 2;6\footnote{\href{https://childes.talkbank.org/browser/index.php?url=Eng-UK/Manchester/Becky/020619.cha}{Manchester/Becky/020619.cha}
}}
\ex \label{3} {\textit{My} get my car. - \textit{my} for \textit{I}, Nina at the age of 2;0\footnote{\href{https://childes.talkbank.org/browser/index.php?url=Eng-NA/Suppes/020003.cha}{Suppes/020003.cha}}}
\end{xlist}
\ex non-accusative object:
\begin{xlist}
\ex \label{2} {Mama put \textit{he} here. - \textit{he} for \textit{him}, Tow at the age of 1;9\footnote{\href{https://childes.talkbank.org/browser/index.php?url=Eng-NA/Post/Tow/010909.cha}{Post/Tow/010909.cha}}}
\ex \label{4} {Watch \textit{my} going round, Mum. - \textit{my} for \textit{me}, Gail at the age of 2;4\footnote{\href{https://childes.talkbank.org/browser/index.php?url=Eng-UK/Manchester/Gail/020421.cha}{Manchester/Gail/020421.cha}}}
\end{xlist}
\ex non-genitive determiner:
\begin{xlist}
\ex \label{5} He doesn't have \textit{them} claws. -\textit{them} for \textit{those/these} Roman at the age of 3;5\footnote{\href{https://childes.talkbank.org/browser/index.php?url=Eng-NA/Weist/Roman/030501.cha}{Weist/Roman/030501.cha}}
\ex \label{6} Um and that's \textit{me} name. - Fraser at the age of 3;0\footnote{\href{https://childes.talkbank.org/browser/index.php?url=Eng-UK/MPI-EVA-Manchester/Fraser/030025a.cha}{MPI-EVA-Manchester/Fraser/030025a.cha}}
\end{xlist}
\end{exe}

The most well-studied error is the non-nominative subject, where an accusative pronoun or a genitive pronoun is used in the subject position, as shown in (\ref{01}). Theoretically, the nominative case is licensed by the +\textsc{finite} feature on the IP. If the +\textsc{finite} is missing, the nominative case won't be checked, and the subject will get the default case, which is the accusative case in English \cite{schutze1997,schutze2001}. Therefore, the syntactic explanation linked this error with children's use of finite verbs \citep[e.g.][]{vainikka1993case,wexler1996, schutze1996subject}. When the children produce a non-finite verb by omitting the inflections, the nominative case will not be checked, leading to the subject getting the default accusative case. Thus, the children would produce errors like `\textit{Him} go to school'. According to the syntactic explanation, when a finite verb is used in the sentence, the subject is unlikely to be a non-nominative pronoun. Although this prediction is generally true for first person pronouns, \cite{pine2005testing} found that the third person singular feminine pronouns do not follow this prediction since the non-nominative form `her' co-occur with finite verbs quite often in children's utterances. 

In addition, the syntactic explanation doesn't address the issues of individual variations, such as why different children make errors with different forms. For example, in example (\ref{1}), Becky replaced the nominative pronoun \textit{he} with its accusative form \textit{him}; whereas Nina replaced the nominative pronoun \textit{I} with the its genitive form \textit{my} in (\ref{3}). Meanwhile, \textit{my} was used as a substitute for the accusative pronoun \textit{me} by Gail in (\ref{4}).  \cite{rispoli1998} studied non-nominative use of pronoun \textit{I} and found that the majority of the children predominantly make `\textit{me}-for-\textit{I}' errors and a minority predominantly make `\textit{my}-for-\textit{I}' errors. He then found that the `\textit{me}-for-\textit{I}' is highly correlated with the correct use of `\textit{me}' in children's production. Thus, \cite{rispoli1998,rispoli1999,rispoli2005} treated pronoun case error as lexical retrieval error that is closely related with the children's pronominal use. He proposed that for each pronoun, case, person, and number form a 3x3x2 paradigm. Young children have difficulties accessing all forms of pronouns in the paradigm, and thus make pronoun errors. 


In addition, parents' input could also play an important role in pronoun errors. \cite{pelham2011input} suggests that English children make more pronoun case errors than German children because English has more case-ambiguous pronouns (e.g. \textit{you}, \textit{it} and \textit{her}) than German. Parents' input also contains many ambiguous phrases like `Let \underline{\textit{me} do} it' or `help \underline{\textit{her} open} it' that could potentially confuse the children, so that they would produce erroneous utterances such as `\underline{\textit{me} do} it' or `\underline{\textit{her} open} it' \citep{tomasello2000, kirjavainen2009can}.  


Apart from non-nominative subjects, children also substitute the accusative case for genitive case, as in (4), and substitute genitive case for an accusative case, as in (5). These errors have been generally overlooked by previous studies.
\begin{exe}
\ex [*]{Don't take \textit{me} bottle out. - \textit{me} for \textit{my}, Nina at the age of 2;1\footnote{\href{https://childes.talkbank.org/browser/index.php?url=Eng-NA/Suppes/020115.cha}{Suppes/020115.cha}}}
\ex [*]{Let \textit{my} try. - \textit{my} for \textit{me}, Carl at the age of 2;6\footnote{\href{https://childes.talkbank.org/browser/index.php?url=Eng-UK/Manchester/Carl/020619.cha}{Manchester/Carl/020619.cha}} }
\end{exe}
The existing explanations seem to fall short for these errors. The syntactic explanations treat the accusative case as the default form. If it is the default, there is no incentive to substitute the accusative case for the genitive case as in (5). Moreover, parents' input never contains any segments that resemble the errors in (4) and (5): `take \textit{me} bottle' and `let \textit{my} try'. 

Moreover, after over 40 years of research, some basic properties of pronoun case errors are still unknown, such as the frequencies and distributions of the errors, at what age  those errors are likely to occur, how prevalent those errors are among children, if the errors disappear after the children reach a certain age or a certain MLU, etc. Few studies have reported comprehensive pronoun error counts or error rates; instead, most of the studies focused on a certain type of errors or pronouns in their analysis. A sizable longitudinal dataset is required to answer these questions. One common limitation for the previous studies is their small sample size: \cite{schutze1996subject} included 3 children (Nina from \citep{suppes1974semantics}, Sarah from \citep{brown1973first} and Peter\citep{bloom1974imitation}); \cite{rispoli1998} collected data on 12 children from 1;0 to 3;0; \cite{kirjavainen2009can} conducted corpus analysis of 17 children in CHILDES (12 from Manchester corpus \cite{theakston2001}, Fraser and Eleanor from densed corpus \citep{rowland2006effect} and Abe from \cite{kuczaj1977acquisition}, Nina from \citep{suppes1974semantics} and Peter from \citep{bloom1974imitation}). Due to limited sample size, previous studies couldn't address these questions. 

This paper investigates pronoun case errors in children's production by conducting a comprehensive corpus analysis on all the available data of English-speaking children on the CHILDES \citep{macwhinney2014childes}. First, this paper reviews some basic features of pronoun case errors, such as frequency and distribution, age range, correlation with age and MLU, etc. Second, this paper re-examines each existing explanation using all available CHILDES data. In the previous studies, researchers either collected their own data or used different corpora to test their hypotheses. It is necessary to evaluate different theories on the same set of data in order to control for collection bias or limitations of small sample size. 


The paper is organized as follows: Section 2 presents a detailed description of pronoun case errors, including the basic properties mentioned above; Section 3 - 5 replicates representative studies for existing explanations and evaluates the explanation; Section 6 discusses the results and implications.