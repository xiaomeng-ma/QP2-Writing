\section{Introduction}
English-speaking children make pronoun case errors from the age of two to four. Some common types of errors include: (1) using an accusative pronoun (e.g. \textit{me, him, us}) or a genitive pronoun (e.g. \textit{my, his, our}) as the subject of the sentence, see (\ref{1}) and (\ref{3}); (2) using a nominative pronoun (e.g. \textit{I, she, he}) or a genitive pronoun as the object in a sentence, see (\ref{2}) and (\ref{4}) and (3) using an accusative pronoun as the determiner, see (\ref{5}) and (\ref{6}). Researchers have categorized such errors as systematic, characteristic and frequent \citep[e.g.][]{huxley1970development, budwig1989linguistic, pelham2011input,fitzgerald2017case}.

\begin{exe}
\ex \label{01} non-nominative subject:
\begin{xlist}
\ex \label{1} {Where does \textit{him} go? - \textit{him} for \textit{he}, Becky at the age of 2;6\footnote{\href{https://childes.talkbank.org/browser/index.php?url=Eng-UK/Manchester/Becky/020619.cha}{Manchester/Becky/020619.cha}
}}
\ex \label{3} {\textit{My} get my car. - \textit{my} for \textit{I}, Nina at the age of 2;0\footnote{\href{https://childes.talkbank.org/browser/index.php?url=Eng-NA/Suppes/020003.cha}{Suppes/020003.cha}}}
\end{xlist}
\ex non-accusative object:
\begin{xlist}
\ex \label{2} {Mama put \textit{he} here. - \textit{he} for \textit{him}, Tow at the age of 1;9\footnote{\href{https://childes.talkbank.org/browser/index.php?url=Eng-NA/Post/Tow/010909.cha}{Post/Tow/010909.cha}}}
\ex \label{4} {Watch \textit{my} going round, Mum. - \textit{my} for \textit{me}, Gail at the age of 2;4\footnote{\href{https://childes.talkbank.org/browser/index.php?url=Eng-UK/Manchester/Gail/020421.cha}{Manchester/Gail/020421.cha}}}
\end{xlist}
\ex non-genitive determiner:
\begin{xlist}
\ex \label{5} He doesn't have \textit{them} claws. -\textit{them} for \textit{those/these} Roman at the age of 3;5\footnote{\href{https://childes.talkbank.org/browser/index.php?url=Eng-NA/Weist/Roman/030501.cha}{Weist/Roman/030501.cha}}
\ex \label{6} Um and that's \textit{me} name. - Fraser at the age of 3;0\footnote{\href{https://childes.talkbank.org/browser/index.php?url=Eng-UK/MPI-EVA-Manchester/Fraser/030025a.cha}{MPI-EVA-Manchester/Fraser/030025a.cha}}
\end{xlist}
\end{exe}

The most well-studied error is the non-nominative subject, where an accusative pronoun or a genitive pronoun is used in the subject position, as shown in (\ref{01}). Theoretically, the nominative case is licensed by the +\textsc{finite} feature on the IP. If the +\textsc{finite} is missing, the nominative case won't be checked. Therefore, the syntactic explanation linked this error with children's use of finite verbs \citep[e.g.][]{vainikka1993case,wexler1996, schutze1996subject}. Young children often go through a period in language development when they sometimes omit inflections on the verbs, producing non-finite forms such as `Then the horse \textit{jump}.' This developmental stage referred to by \cite{wexler1994,wexler1998,wexler2000} as the `Optional Infinitive Stage', where finiteness is optional in children's grammar, even though they already have the syntactic knowledge of tense and agreement. When the children produce a non-finite verb by omitting the inflections, the nominative case are not checked, leading to the non-nominative subject errors, such as `Then \textit{him/his} jump'. In the syntactic explanation, the non-nominative subject stems from children's problematic use of finite verbs and it is characteristic at Optional Infinitive Stage.

Although the syntactic explanation can account for many observed error patterns, not all the predictions are correct. According to the syntactic explanation, when a non-nominative subject is present, the verb in that sentence is unlikely to be a finite verb. \cite{pine2005testing} found that the data doesn't always support this prediction since it is not uncommon for children to produce non-nominative subjects with finite verbs. 

In addition, the syntactic explanation doesn't address the questions of individual variations, such as why different children make errors with different forms. For example, in example (\ref{1}), Becky replaced the nominative pronoun \textit{he} with its accusative form \textit{him}; whereas Nina replaced the nominative pronoun \textit{I} with the its genitive form \textit{my} in (\ref{3}). Meanwhile, \textit{my} was used as a substitute for the accusative pronoun \textit{me} by Gail in (\ref{4}).  \cite{rispoli1998} studied non-nominative use of pronoun \textit{I} and found that the majority of the children predominantly make `\textit{me}-for-\textit{I}' error and a minority predominantly make `\textit{my}-for-\textit{I}' error. He then found that the `\textit{me}-for-\textit{I}' is highly correlated with the correct use of `\textit{me}' in children's production. Thus, \cite{rispoli1998,rispoli1999,rispoli2005} treated pronoun case error as lexical retrieval error that is closely related with children's pronominal use. He proposed that for each pronoun, case, person, and number form a 3x3 paradigm. Young children have difficulties accessing all forms of pronouns in the paradigm, and thus make pronoun errors. 




In addition, parents' input also could play an important role in pronoun errors. \cite{pelham2011input} suggests that English children make more pronoun case errors than German children because English has more case-ambiguous pronouns (e.g. \textit{you}, \textit{it} and \textit{her}) than German. Parents' input also contains many ambiguous phrases like `Let \underline{\textit{me} do it}' or `help \underline{\textit{her} open it}' that could potentially confuse the children, that they would produce erroneous utterances such as `\underline{\textit{me} do it}' or `\underline{\textit{her} open it}' \citep{tomasello2000, kirjavainen2009can}.  




Apart from non-nominative subjects, children also substitute the accusative case for genitive case, as in (4) and substitute genitive case for an accusative case, as in (5). These errors have been generally overlooked by previous studies.
\begin{exe}
\ex [*]{Don't take \textit{me} bottle out. - \textit{me} for \textit{my}, Nina at the age of 2;1\footnote{\href{https://childes.talkbank.org/browser/index.php?url=Eng-NA/Suppes/020115.cha}{Suppes/020115.cha}}}
\ex [*]{Let \textit{my} try. - \textit{my} for \textit{me}, Carl at the age of 2;6\footnote{\href{https://childes.talkbank.org/browser/index.php?url=Eng-UK/Manchester/Carl/020619.cha}{Manchester/Carl/020619.cha}} }
\end{exe}
The existing explanations seem to fall short for these errors. The syntactic explanations treat the accusative case and the genitive case as two different default forms to the nominative case. If both cases are default, there is no incentive to substitute one default for the other as in (4) and (5). Moreover, parents' input never contain any segments that resemble the errors in (4) and (5): `take \textit{me} bottle' and `let \textit{my} try'. 

Moreover, after over 40 years of research, some basic properties of pronoun case errors are still unknown, such as the frequencies and distributions of the errors, at what age are those errors likely to occur, how prevalent are those errors among children, does the error disappear after the children reach a certain age or a certain MLU, etc. Few studies have reported comprehensive pronoun error counts or error rates; instead, most of the studies focused on a certain type of errors or pronouns in their analysis. A sizable longitudinal dataset is required to answer these questions. One common limitation for previous studies is their small sample size: \cite{schutze1996subject} included 3 children (Nina from \citep{suppes1974semantics}, Sarah from \citep{brown1973first} and Peter\citep{bloom1974imitation}); \cite{rispoli1998} collected data on 12 children from 1;0 to 3;0; \cite{kirjavainen2009can} conducted corpus analysis of 17 children in CHILDES (12 from Manchester corpus \cite{theakston2001}, Fraser and Eleanor from densed corpus \citep{rowland2006effect} and Abe from \cite{kuczaj1977acquisition}, Nina from \citep{suppes1974semantics} and Peter from \citep{bloom1974imitation}). Due to limited sample size, previous studies couldn't address these questions. 

This paper investigated pronoun case errors in children's production by conducting a comprehensive corpus analysis on all the available data of English-speaking children on \textsc{childes} \citep{macwhinney2014childes}. First, this paper reviewed some basic features of pronoun case errors, such as frequency and distribution, age range, correlation with age and MLU, etc. Second, this paper re-examined each existing explanations using all available CHILDES data. In the previous studies, researchers either collected their own data or used different corpora to test their hypotheses. It is necessary to evaluate different theories on the same set of data in order to control for collection bias or limitations of small sample size. 

Last, this paper will provide a hypothesis about how children acquire pronoun case in the first place. Cases are used to mark different relationship between arguments, which is a more abstract grammatical feature to learn compared to plural forms or tense marking. Yet young children are able to produce the correct form of a pronoun in different argument positions. It would be worthwhile to ask the question: how do children acquire this abstract feature at such a young age? This study is going to test two hypotheses of the acquisition of pronoun case. From a theoretical perspective, the children could derive different cases through argument structure. Nominative case and accusative case could be differentiated at the sentential level, such that the former is associated with the subject and the latter case is associated with the object. In addition, children could also acquire cases through statistical learning. Nominative case, accusative case and genitive case have different distributional patterns in speech, which difference could be utilized by children to acquire different cases. 

The paper is organized as follows: Section 2 presents a detailed description of pronoun case errors, including the basic properties metioned above; Section 3 - 5 replicates representative studies for existing explanations and evaluates that explanation; Section 6 demonstrates a computational model using distributional patterns to explain pronoun case acquisition.